\documentstyle[12pt]{report}

\topmargin -0.25in
\headheight 10pt
\headsep 30pt
\footheight 24pt
\oddsidemargin 0.0in
\evensidemargin 0.0in
\parskip 4mm
\footskip 50pt
\textheight 8.5in
\textwidth 6.5in

%%%%%%% changebar.sty %%%%%%%
\makeatletter
% Change bar document-style option for LaTeX.
% Copyright (C) 1990 by David B. Johnson (dbj@titan.rice.edu).
 
% These macros draw a solid bar down the right margin of the output,
% covering a range of the input file that has been declared to be changed.
%
% The beginning and end of a change bar in the text are marked with
% \chgbarbegin and \chgbarend, respectively.  For example,
%
%     Here is some sample text \chgbarbegin that was
%     changed\chgbarend{} and some that wasn't changed.
%
% The change bar is drawn continuously between the line of output
% containing the \chgbarbegin and the line of output containing the
% \chgbarend.  These lines can end up on separate pages, and the
% division at page boundaries is handled automatically.
 
% Two dimensions control the size and placement of the change bars:
%     \chgbarwidth      The width of a change bar
%     \chgbarsep        The distance between the text and the change bar
 
% Warning: it does not appear to be possible to do this completely
% correctly, due to the time at which the verticle glue on a page is
% finally set, and the way that page breaks are decided.  With
% \raggedbottom, this normally works fine.  It hasn't been tested with
% \flushbottom, but will probably behave worse.  In strange rare
% situations, a change bar might be drawn from the first line of a page
% up off the top of a page; this can usually be fixed by slightly moving
% the \chngbarend around, or by breaking a single change bar range
% into two ranges.
 
\newdimen\chgbarwidth \newdimen\chgbarsep
\chgbarwidth 4pt
\chgbarsep .25in
 
\def\chgbarbegin{\ifhmode\@chgbar{-2}\else\@chgbar{-3}\fi}
\def\chgbarend{\@chgbar{-4}}
 
\marginparpush 0pt
 
% The remainder of this is hacked up based on the LaTeX 2.09 latex.tex.
 
% copied from \marginpar
\def\@chgbar#1{\ifhmode \@bsphack\@floatpenalty -\@Mii\else
   \@floatpenalty-\@Miii\fi\ifinner
      \@parmoderr\@floatpenalty\z@
    \else\@next\@currbox\@freelist{\global
      \count\@currbox#1}{\@floatpenalty\z@ \@fltovf
      \def\@currbox{\@tempboxa}}\fi
     \setbox\@tempboxa\vbox
     \bgroup\end@float\@esphack}
 
\newdimen\@chgbarbegin
\newif\if@inchgbar \@inchgbarfalse
 
\def\@addmarginpar{%
\ifnum\count\@currbox = -2	% change bar begin from hmode
    \global\@chgbarbegin\@pageht \global\advance\@chgbarbegin -\baselineskip
    \global\@inchgbartrue
    \@cons\@freelist\@currbox
\else
\ifnum\count\@currbox = -3	% change bar begin not from hmode
    \global\@chgbarbegin\@pageht
    \global\@inchgbartrue
    \@cons\@freelist\@currbox
\else
\ifnum\count\@currbox = -4	% change bar end
    \if@inchgbar\else\@latexbug\fi
    \@tempdima\@pageht \advance\@tempdima -\@chgbarbegin
    \nointerlineskip
    \@tempcnta\@ne
    \if@twocolumn
        \if@firstcolumn \@tempcnta\m@ne \fi
    \else
      \if@mparswitch
         \ifodd\c@page \else\@tempcnta\m@ne \fi
      \fi
      \if@reversemargin \@tempcnta -\@tempcnta \fi
    \fi
    \hbox to\columnwidth
      {\ifnum \@tempcnta >\z@
          \hskip\columnwidth \hskip\chgbarsep
        \else \hskip -\chgbarsep \fi
\hbox{\vbox to 0pt{\vss
       \hrule \@height\@tempdima \@width\chgbarwidth \@depth\z@
}}
\hss}
    \nointerlineskip
    \global\@inchgbarfalse
    \@cons\@freelist\@currbox
\else
    \@next\@marbox\@currlist{\@cons\@freelist\@marbox
    \@cons\@freelist\@currbox}\@latexbug\@tempcnta\@ne
    \if@twocolumn
        \if@firstcolumn \@tempcnta\m@ne \fi
    \else
      \if@mparswitch
         \ifodd\c@page \else\@tempcnta\m@ne \fi
      \fi
      \if@reversemargin \@tempcnta -\@tempcnta \fi
    \fi
    \ifnum\@tempcnta <\z@  \global\setbox\@marbox\box\@currbox \fi
    \@tempdima\@mparbottom \advance\@tempdima -\@pageht
       \advance\@tempdima\ht\@marbox \ifdim\@tempdima >\z@
       \@warning{Marginpar on page \thepage\space moved}\else\@tempdima\z@ \fi
    \global\@mparbottom\@pageht \global\advance\@mparbottom\@tempdima
       \global\advance\@mparbottom\dp\@marbox
       \global\advance\@mparbottom\marginparpush
    \advance\@tempdima -\ht\@marbox
    \global\ht\@marbox\z@ \global\dp\@marbox\z@
    \vskip -\@pagedp \vskip\@tempdima\nointerlineskip
    \hbox to\columnwidth
      {\ifnum \@tempcnta >\z@
          \hskip\columnwidth \hskip\marginparsep
        \else \hskip -\marginparsep \hskip -\marginparwidth \fi
       \box\@marbox \hss}
    \vskip -\@tempdima
    \nointerlineskip
    \hbox{\vrule \@height\z@ \@width\z@ \@depth\@pagedp}
\fi\fi\fi}
 
\def\@makecol{\setbox\@outputbox\box\@cclv
   \if@inchgbar
    \@tempcnta\@ne
    \if@twocolumn
        \if@firstcolumn \@tempcnta\m@ne \fi
    \else
      \if@mparswitch
         \ifodd\c@page \else\@tempcnta\m@ne \fi
      \fi
      \if@reversemargin \@tempcnta -\@tempcnta \fi
    \fi
    \@tempdima\ht\@outputbox \advance\@tempdima -\@chgbarbegin
    \advance\@tempdima -\baselineskip
    \setbox\@outputbox
     \vbox{\boxmaxdepth \maxdepth
     \unvbox\@outputbox \nointerlineskip \hbox to\columnwidth
      {\ifnum \@tempcnta >\z@
          \hskip\columnwidth \hskip\chgbarsep
        \else \hskip -\chgbarsep \fi
       \hbox{\vbox to 0pt{\vss
         \hrule \@height\@tempdima \@width\chgbarwidth \@depth\z@}}\hss}}
    \global\@chgbarbegin 0pt
\fi
   \ifvoid\footins\else\setbox\@outputbox
     \vbox{\boxmaxdepth \maxdepth
     \unvbox\@outputbox\vskip\skip\footins\footnoterule\unvbox\footins}\fi
     \xdef\@freelist{\@freelist\@midlist}\gdef\@midlist{}\@combinefloats
     \setbox\@outputbox\vbox to\@colht{\boxmaxdepth\maxdepth
        \@texttop\dimen128=\dp\@outputbox\unvbox\@outputbox
        \vskip-\dimen128\@textbottom}
     \global\maxdepth\@maxdepth}
 
\makeatother
%%%%%%% end changebar.sty %%%%%%%

\def\CLPR{\mbox{CLP(${\cal R}$)}}
\def\cut{\mbox{!}}

\newcommand{\indexentry}[2]{\protect\item{\protect#1}, #2}
\newcommand{\nl}{\hfill \\} % silent line break
\newcommand{\ignore}[1]{}
\newcommand{\ouritem}[1]{\item[{\mbox{\tt #1}}]\ \\}
\newcommand{\oursitem}[1]{\item[{\mbox{\tt #1}}]\ \\\vspace{-0.33in}}
\newcommand{\unint}[1]{ \mbox{ $\widehat{#1}$ } }
\newcommand{\unintt}[1]{ \mbox{$\widehat{\mbox{\tt #1}}$} }
\newcommand{\p}[1]{ \mbox{{\tt #1}} }
\newcommand{\pct}{\mbox{\%}}

\newcommand{\lhs}{\frenchspacing l.h.s. }
\newcommand{\rhs}{\frenchspacing r.h.s. }
\newcommand{\wrt}{\frenchspacing w.r.t. }
\newcommand{\singlespace}{
	\renewcommand{\baselinestretch}{1.00} \large \normalsize}
\newcommand{\doublespace}{
	\renewcommand{\baselinestretch}{1.75} \large \normalsize}
\newcommand{\ttsl}{$\tt \backslash$}

\newcommand{\stuff}[1]{
	\begin{minipage}{4in}
	{\tt \samepage 
	\begin{tabbing} 
	\hspace{8mm} \= \hspace{6mm} \= \hspace{10mm} \= \hspace{55mm} \= \kill
	#1
	\end{tabbing}
	}
	\end{minipage}
}

\makeindex

\pagestyle{empty}
\begin{document}
\title{\bf The \CLPR\ Programmer's Manual \\ Version 1.2}
\author{
{\sc Nevin C. Heintze} \ $^{\dag}$ \\
{\sc Joxan Jaffar} \ $^{\ddag}$ \\ 
{\sc Spiro Michaylov} \ $^{*}$ \\
{\sc Peter J. Stuckey} \ $^{\S}$ \\
{\sc Roland H.C. Yap} \ $^{\P\S}$ \\ \ \\
$^{\ddag}$ 
{\normalsize\em IBM Thomas J Watson Research Center} \\
{\normalsize\em PO Box 704} \\
{\normalsize\em Yorktown Heights, NY 10598, U.S.A.}
\\ \ \\
$^{\dag}$ 
{\normalsize\em School of Computer Science} \\
{\normalsize\em Carnegie Mellon University} \\
{\normalsize\em Pittsburgh, PA 15213, U.S.A.}
\\ \ \\
$^{*}$
{\normalsize\em Department of Computer and Information Science} \\
{\normalsize\em The Ohio State University} \\
{\normalsize\em Columbus, OH 43210-1277, U.S.A.}
\\ \ \\
$^{\S}$ 
{\normalsize\em Department of Computer Science} \\
{\normalsize\em University of Melbourne} \\ 
{\normalsize\em Parkville, Victoria 3052, Australia}
\\ \ \\
$^{\P}$ 
{\normalsize\em Department of Computer Science} \\
{\normalsize\em Monash University} \\ 
{\normalsize\em Clayton, Victoria 3168, Australia}
%\\ \ \\ 
}
\date{September 1992}
\maketitle
\pagestyle{headings}

%\newpage
\pagenumbering{roman}
\tableofcontents

\chapter{Introduction}
\pagenumbering{arabic}
This manual describes \CLPR{} version 1.2, and at a number of places
throughout this text changebars have been placed either to indicate new
features in version 1.2 from version 1.1 or some changes in the manual.
The \CLPR\ language is an instance of the Constraint Logic Programming scheme
defined by Jaffar and Lassez \cite{JAFFAR87A}.
Its operational model is similar to that of
PROLOG. A major difference is that unification is replaced by a more general
mechanism: solving constraints in the domain of uninterpreted functors over
real arithmetic terms. A working knowledge of PROLOG programming is assumed in
this document; the book by Sterling and Shapiro \cite{STERLING86} 
can serve as a suitable introductory text.  Further technical
information on \CLPR\  is available on language design and implementation
\cite{JAFFAR87B,JAFFAR92C}, meta-programming \cite{HEINTZE89} and delay
mechanisms \cite{JAFFAR91}. 
Additionally, much has been written about
applications in electrical engineering \cite{HEINTZE87,MOZETIC91}, 
differential equations \cite{HARLAND87,HOMIAK91}, 
temporal reasoning \cite{AMON92,AMON92B,BRZOSKA91},
protocol testing \cite{GORLICK87},
structural analysis and synthesis \cite{LAKMAZAHERI89},
mechanical engineering \cite{STHANUSUBRAMONIAN91},
user interfaces \cite{YEUNG88},
model-based diagnosis \cite{YU91},
options trading \cite{LASSEZ87A}, 
music theory \cite{TOBIAS88},
molecular biology \cite{YAP91},
{\em etc}.

This document is both an introductory tutorial and  
reference manual describing the compiler-based
implementation of \CLPR. The reader experienced with PROLOG or \CLPR{}
may wish to 
skip to Chapter \ref{chap:system}, and in particular, see the sample
session in Section \ref{sample-session} to get started quickly.
Compiled \CLPR\ is an interactive system that compiles all programs and
goals into CLAM code which is interpreted by a byte-code emulator
that is part of the system. The system is portable in the sense that it will
run on virtually all 32 bit UNIX$^{\mbox{\scriptsize TM}}$ 
machines with a reasonably 
standard C compiler, as well as many others.

We would like to emphasize that this manual describes a constantly-evolving,
experimental system. Hence much of what is described is subject to change in
future releases. Furthermore, the use of undocumented features is 
particularly dangerous.

Finally, we adopt some standard notational conventions, 
such as the name/arity convention for describing predicates and functors, 
{\tt +} for input arguments, {\tt -} for output arguments, and {\tt ?} for 
arguments that may be either input or output.\index{notation conventions}

\chapter{Syntax and Simple Examples} \label{syntax-examples}

A \CLPR\ program is a collection of rules. 
The definition of a
rule is similar to that of a PROLOG clause, but it differs in two 
important ways: rules can contain constraints as well as atoms in the body,
and the definition of terms is more general. A goal is a rule without a head,
as usual.\index{program}\index{rule}\index{goal}

The body of a rule may contain any number of arithmetic
constraints, separated by commas in the usual way.
Constraints are equations or inequalities, built up from real
constants, variables, {\tt +}, {\tt -}, {\tt *}, {\tt /}, and 
{\tt =}, {\tt >=}, {\tt <=}, {\tt >}, {\tt <}
where all of these symbols have the usual meanings and parentheses
may be used in the usual way to resolve ambiguity.  Unary arithmetic 
negation is also available, as are some special
interpreted function symbols which will be described later.
Any variable that
appears in an arithmetic constraint is said to be an arithmetic variable, 
and cannot take a non-arithmetic value. These constraints may be thought of as
built-in predicates written infix, but they are really much more powerful, as
we shall see later. Goals are also similar to those in PROLOG, and may
contain explicit constraints as well.

Comments in the program are either in the PROLOG style,
beginning with a ``{\tt \%}'' and continuing to the
end of the line, or
also in the form of C style comments, starting with ``{\tt /*}''
and ending with ``{\tt */}'' (comments can contain newlines).
Unlike normal C comments, these can be nested so that
code already containing comments can be commented easily. \index{comments}

\section{Terms and Constraints}

Syntactically, 
a term is either a simple term or a compound term constructed from
simple terms.
A term is then either an arithmetic term or a functor term.
The simple terms are:
\begin{itemize}
\item Variable terms \\
A variable is a sequence of alphanumeric
characters (including ``{\tt \_}''),
either begins with an uppercase alphabetic character or an
underscore ``{\tt \_}''. Variables consisting of an
underscore only are anonymous variables and always represent a new variable.
Variables that are longer than one character and begin with an underscore
are the same as any other ordinary variable,\footnote{ 
These are not anonymous variables.} except that they are ignored
for the purposes of style checking.

\item Numeric constant terms\\
\chgbarbegin
This is a real number with an optional decimal point and optional integer
exponent which may be positive or negative.
\chgbarend

\item Symbolic numeric constants\\
\chgbarbegin
These denote special constant values, eg. $\pi$ and have the
syntax {\tt \#}$<\!name\!>$ where the name is just an
atomic functor constant. The following are the special constants
defined by default:

\begin{tabular}{lrll}
{\tt \#p} & $\pi = $ & $ 3.14159265358979323846 $ \\
{\tt \#p\_2} & $\pi/2 = $ & $ 1.57079632679489661923 $ \\
{\tt \#p\_4} & $\pi/4 = $ & $ 0.78539816339744830962 $ \\
{\tt \#e} & $e = $ & $ 2.7182818284590452354 $ \\
{\tt \#sqrt2} & $\sqrt{2} = $ & $ 1.41421356237309504880 $ \\
{\tt \#sqrt1\_2} & $1/\sqrt{2} = $ & $ 0.70710678118654752440 $ \\
{\tt \#c} & $c = $ & $ 2.99792458*10^8$ & (speed of light in vacumn) \\
{\tt \#g} & $g = $ & $ 9.80665$ & (acceleration of gravity) \\
{\tt \#h} & $h = $ & $ 6.626176*10^{-34}$ & (Planck's constant) \\
{\tt \#ec} & & $1.6021892*10^{-19}$ & (elementary charge) \\
\end{tabular}

\noindent
There are also some handy metric conversion ratios predefined:

\begin{tabular}{lll}

{\tt \#cm2in} & 0.393701 & (centimeters to inches)\\
{\tt \#km2mile} &  0.62137 & (kilometers to miles)\\
{\tt \#gm2oz} & 0.03527 & (grams to ounces)\\
{\tt \#kg2lb} & 2.20462 & (kilograms to pounds)\\
{\tt \#l2gal} &  0.21998 & (litres to imperial gallons)\\
{\tt \#l2usgal} & 0.26418 & (litres to US gallons)\\
\end{tabular}

\noindent
(Note that new constants can be created by using 
{\tt new\_constant/2}.)
\chgbarend

\item functor constant terms\\
These are either a sequence of alphanumeric characters (including ``{\tt \_}'',
starting with a lowercase letter; or a sequence of characters from the set,
\[\left\{ \rule[0.5ex]{0pt}{1.5ex} \mbox{\tt \&*+-./:;<=>?@{\\}\^{}\~{}} \right\}\]
Also any sequence of characters delimited by single quotes ``{\tt '}''
is allowed, e.g. {\tt 'foo + bar'} is a functor constant (atom) with
that name including the blanks.
The special constant ``{\tt []}'' denotes the empty list or nil.
Note also that the special arithmetic function symbols, though having
the same syntax, are arithmetic terms and not functor terms.

\item String constant terms\\
This is any sequence of characters delimited by double quotes ({\tt "}).
NOTE: At present the interpretation of strings in the syntax has not
been finalized and all strings are being treated as functor constants
(i.e. the single quote form).
This differs from some PROLOG's which use this syntax
as an alternative notation for lists.

\end{itemize}

\chgbarbegin
An arithmetic term is either a variable, numeric constant 
\index{arithmetic term}
or a compound term built up from arithmetic terms in the usual way
using the arithmetic function symbols:
{\tt +}, {\tt -}, {\tt *}, {\tt /}, 
{\tt sin}, {\tt arcsin}, {\tt cos}, {\tt arccos},
{\tt pow}, {\tt abs}, {\tt min} and {\tt max}.
\chgbarend
For example,
 
\stuff{
\> X \\
\> 3.14159 \\
\> 42e-8 \\
\> X + Y \\
\> sin(X + 2.0) \\
\> (X + Y) / 4 
}

\noindent
are all valid arithmetic terms. 
However, 

\stuff{
\> f(a) \\
\> c + 5.0 \\
\> cos(f(3))
}

\noindent
are not.
The arithmetic terms are interpreted as having their usual meaning
as arithmetic expressions.
Operator precedences for the arithmetic
function symbols follow the normal convention\footnote{
User defined unary or binary operators in the standard PROLOG
fashion using {\tt op/3} are also supported.
}.
Parentheses can be also used to escape the application of
the default operator precedences.

Functor terms are either variable or functor \index{functor term}
constant terms or compound terms.
A compound functor term has the form
$f(t_1, t_2, \cdots , t_N)$ where $N \geq 0$, $f$ is
an $N$-ary uninterpreted functor and 
$t_1, t_2, \cdots , t_N$ are (not necessarily functor) terms.
The functor is uninterpreted, meaning that the functor
is simply to be treated as a symbolic constant,
as opposed to the arithmetic terms, which are interpreted.
The allowable syntax of the functor symbol $f$
is that of any functor constant term.
The other compound functor terms are lists, which are specified using
the usual PROLOG list notation ({\tt [L]}), for example ``{\tt [a, b]}''.
A dot notation for lists, as in ``{\tt a.b.[]}'', may also be used.
For example, the following are valid terms:

\stuff{
\> [a, 1+X] \\
\> f([3.12, g(a)]) \\
\> f(c) \\
\> f(X) \\
\> f(3.14159) \\
\> g(22, h(4))  \\
\> f(X + 3)
}

\index{constraint}\index{arithmetic constraint}\index{functor constraint}
\noindent
~~A {\em constraint} is either an {\em arithmetic constraint} or
a {\em functor constraint}.  The former is defined to be of the
the form $t_1 ~\Delta~ t_2$ where $t_1$ and $t_2$ are arithmetic terms
and $\Delta$ is one of the arithmetic relations 
{\tt =}, {\tt >=}, {\tt <=}, {\tt >}, and {\tt <}.
For example, 

\stuff{
\> X > 5.0 \\
\> X + Y + Z = 3 \\
\> X <= Y \\
\> X = V \\
\> 3 = sin(X) \\
\> 1.234 + X < Y 
}

\noindent
are all valid arithmetic constraints, while the following are not.

\stuff{
\> c > Y \\
\> X = 3.0 < Y  \\
\> pow(X = Y, 3) \\
\> 4 < X < 5
}

\noindent
A functor constraint is of the form $t_1 = t_2$ where
each of $t_1$ and $t_2$ is either a variable or a functor term.
We shall sometimes refer to a functor constraint as a functor equation below.

\section{Some Simple Programs}

Now we will look at some example programs without considering 
the details of their execution.
The first example is a program expressing the relation {\tt fib(N, X)} where 
{\tt X} is the {\tt N}th Fibonacci number. 

\stuff{
\> fib(0, 1). \\
\> fib(1, 1). \\
\> fib(N, X1 + X2) :- \\
\>\> 	N > 1, \\
\>\> 	fib(N - 1, X1), \\
\>\> 	fib(N - 2, X2).
}

\noindent
To compute the 10th Fibonacci number, 
we can use the goal

\stuff{
\> ?- fib(10, Z).
}

\noindent
while to find out which Fibonacci number is 89, we can use the goal

\stuff{
\> ?- fib(X, 89).
}

\noindent
The next program describes the relationship between two complex numbers and
their product. We will represent the complex number {\tt X + iY} as 
the term {\tt c(X, Y)}.

\stuff{
\> zmul(c(R1, I1), c(R2, I2), c(R3, I3)) :- \\
\>\>	R3 = R1 * R2 - I1 * I2 , \\
\>\>	I3 = R1 * I2 + R2 * I1 .
}

\noindent
Any of the following goals will return a unique answer. The first goal asks
for the product of two complex numbers, while the other two ask for the
result when one complex number is divided by another. 

\stuff{
\> ?- zmul(c(1, 1), c(2, 2), Z). \\
\> ?- zmul(c(1, 1), Y, c(0, 4)). \\
\> ?- zmul(X, c(2, 2), c(0, 4)).
}

\noindent
Notice how both
operations are described using the definition of complex multiplication,
rather than writing a separate rule that divides complex numbers by first
realizing the divisor and then multiplying. This declarative aspect will be an
important
feature of many of the programs we look at. Also notice that both of the 
programs we have seen so far have been invertible in the sense that it did
not matter which terms in the goals were ground and which were not. This is a
property that we will try to obtain as often as possible when we define
programs or parts of programs. 
As a further example,
the special {\tt pow} function can 
be used to compute powers, roots and logarithms
of an arbitrary base. The rules below for square root,

\stuff{
\> sqroot(X, pow(X, 0.5)):- \\
\>\>	X >= 0. \\
\> sqroot(X, -pow(X, 0.5)) :- \\
\>\>	X >= 0.
}

\noindent
state that a non-negative number has a positive and negative square root.
Finally consider the following program, which relates 
the key parameters in a mortgage.

\stuff{
\> mortgage(P, Time, IntRate, Bal, MP) :- \\
\>\>   Time > 0, Time <= 1, \\
\>\>   Bal = P * (1 + Time * IntRate/1200) - Time * MP. \\
\> mortgage(P, Time, IntRate, Bal, MP) :- \\
\>\>   Time > 1, \\
\>\>   mortgage(P*(1 + IntRate/1200) - MP, Time-1, IntRate, Bal, MP).
}

\noindent
The parameters above are principal, life of the mortgage (in months),
annual interest rate (\%) which is compounded monthly,
the monthly payment, and finally, the outstanding balance.
The goal

\stuff{
\> ?- mortgage(100000, 180, 12, 0, MP).
}

\noindent
asks the straightforward query as to how much it would cost
to finance a \$100,000 mortgage at $12$ percent for $15$ years,
and the answer obtained is {\tt MP = 1200.17}.
We can ask the question backwards:

\stuff{
\> ?- mortgage(P, 180, 12, 0, 1200.17).
}

\noindent
to obtain the expected answer {\tt P = 100000}, or ask for how long
a mortgage is needed:

\stuff{
\> ?- mortgage(100000, Time, 12, Bal, 1300).
}

\noindent
Here we get the answer {\tt Time = 147.365}.
The main point of this example, however, is that we can ask,
not for the values of, but for the {\em relationship between\/}
{\tt P}, {\tt MP} and {\tt Bal}.  For example,

\stuff{
\> ?- mortgage(P, 180, 12, Bal, MP).
}

\noindent
gives the answer

\stuff{
\> P = 0.166783 * Bal + 83.3217 * MP
}

\noindent
This particular example illustrates how
answer constraints may be viewed as a partial evaluation
of the program.  In this case, the equation above is the result of
partially evaluating the program with respect to
{\tt Time = 180} and {\tt I = 12}.

\section{The Type Issue}\index{type}

Informally, one of the two types in \CLPR\ is given by the real numbers, 
and the other by the remaining ground (variable-free) terms.  
Strictly speaking, \CLPR\ is a statically typed language in the sense
that variables, uninterpreted functors and predicates in a program
must be used in a consistent way with respect to their type.
That is, each variable and each argument of 
every predicate and uninterpreted functor 
is first acknowledged to be of a certain type.
The program is then considered to be ill-typed if, for example,
a variable appears both in a functor constraint and an arithmetic
constraint; similarly, the program is ill-typed  
if one occurrence of a predicate or uninterpreted functor 
has a number in the first argument while, in another occurrence,
it has a functor term in the first argument.

For programming convenience, however, \CLPR\ does not perform
such type-checking at compile time.  This decision is based on
the fact that it is often useful to overload a symbol; for example,
one may want a database {\tt p} of both numbers and letters:

\begin{minipage}{4in}
{\tt \samepage
\begin{tabbing}
\hspace{10mm} \= \hspace{10mm} \= \hspace{10mm} \= \kill
\> p(1). \\
\> p(2). \\
\> p(a). \\
\> p(b).
\end{tabbing}
}
\end{minipage}

\noindent
and one may run a goal containing {\tt p(X)} and some
constraints used for selection within the database.
Note that by not performing type-checking, one can have a
runtime type error. That is, an execution sequence which 
fails because of a ``type clash''.
Often such failures indicate that there is an error in the program.
The \CLPR\ system will not distinguish such failures from failures obtained
from well-typed constraints.

A straightforward way of thinking about the type issue when
writing \CLPR\ programs is that whenever an arithmetic term
appears in a rule, for each variable {\tt X} therein,
we can implicitly add a corresponding atom {\tt real(X)} 
to the body of the rule.
The system predicate {\tt real/1} is {\em true} just in case
there is a real solution for {\tt X} in the context of the current
collection of constraints.

\chapter{Programming in \CLPR}

\section{Preliminaries}

Before we can look at more advanced programming examples, it is necessary to
have some idea of how the programs are executed. This is similar in flavor
to the way PROLOG programs are executed, but the basic operational 
step of unifying an atom with the head of a rule is
replaced by something more general.
In this preliminary section, we assume that all arithmetic constraints
are {\em linear}; the general case is discussed in a later section.

The computation begins with a goal and an initially empty set of
{\em collected constraints}.
The usual left-right atom selection rule is used to select either an
arithmetic constraint or an atom at each stage. 
When a constraint is selected, 
it is added to the set of collected constraints,
and it is determined whether the resulting set has a solution.
If there is no solution, backtracking takes place in the usual way.
On the other hand, when an atom is selected, the set of 
rules is searched in the usual top-down fashion, each time 
matching that atom with the head of some rule.
Such a match is realized by an equation between these two atoms;
such an equation is treated like any equation between terms.

As before, it is required that the system of constraints collected 
so far has a solution.
In general, solving this equation proceeds at first by unifying the
syntactic parts of the terms in the usual way. However, these terms may
contain arithmetic terms. As arithmetic terms have a special meaning, 
they are not unified syntactically, but rather an equation between
them is solved in the domain of real arithmetic. 

Let us consider some examples. 
We start with a program that has no explicit
constraints or arithmetic terms, effectively written in PROLOG.

\stuff{
\> p(f(c)). \\
\> q(g(X)) :- \\
\>\>	p(f(X)). \\
\> ?- q(Y).
}

As the computation proceeds, the collected constraint set and current goal
are as follows:

\stuff{
\> \{\} ?- q(Y). \\
\> \{q(Y) = q(g(X)) \} ?- p(f(X)). \\
\> \{q(Y) = q(g(X)), p(f(X)) = p(f(c)) \} ?- .
}

\noindent
Note that only one successful path is shown here. 
Also, as we will discuss in more detail later, the ``answer'' 
to this query is just the set of constraints
collected, but ``projected'' onto the goal variables, in this case {\tt Y}. 
So the answer to the above query is 

\stuff{
\> Y = g(c).
}

\noindent
Now consider a program that includes both arithmetic terms
and explicit constraints:

\stuff{
\> p(10, 10). \\
\> q(W, c(U, V)) :- \\
\>\>	W - U + V = 10, \\
\>\>	p(U, V). \\
\> ?- q(Z, c(X + Y, X - Y)).
}

\noindent
and again we only look at one successful path of the execution:

\stuff{
\> \{\} ?- q(Z, c(X + Y, X - Y)). \\
\> \{q(Z, c(X + Y, X - Y)) = q(W, c(U, V)) \} ?- W - U + V = 10, p(U, V). \\
\> \{q(Z, c(X + Y, X - Y)) = q(W, c(U, V)), W - U + V = 10 \} ?- p(U, V). \\
\> \{$\cdots$ , p(U,V) = p(10, 10) \} ?- .\\
}

\noindent
The answer for this derivation is

\stuff{
\> Y = 0, X = 10, Z = 10.
}

\noindent
and we should notice that, as expected,
it does not contain any mention of the variables U, V, and W. 
Also note that, in general, the answers need not give values to 
variables, and it is possible to get an answer constraint like

\stuff{
\> X + Y + Z = 0, X > Y.
}

\noindent
This facility is a very important and useful feature of \CLPR\ as we will
illustrate later.

\section{Delay of Nonlinear Constraints} \index{delayed constraint}
\index{nonlinear constraint}

In the above discussion of the operational model, we saw how each
operational step results in one or more constraints being added to the
collected constraint set, and the new set being checked for satisfiability.
Because of efficiency requirements, there
is a limit to how sophisticated the decision algorithm for constraints
can be, and consequently the collected constraint set may get too
complicated for the decision algorithm. In particular, consider a case when
the collected constraint set is solvable, but one constraint is added
that makes the set so complicated that it is not practical to decide
whether it has remained solvable. 

A naive approach to dealing with this problem is simply
to disallow expressions that can result in such complexity. 
This is tantamount to disallowing all nonlinear constraints.
The loss in expressive power is, however, unacceptable.
Instead, \CLPR{} allows nonlinear constraints but keeps them in a 
{\em delayed constraint set}. 
More precisely, at each operational
step, instead of blindly adding each constraint to the collected constraint
set and incurring the cost of performing a satisfiability test, 
we remove certain constraints that would make the set too complicated.
We keep these removed constraints in the delayed constraint set. 
Additionally, at each step it is
possible that some constraint in the delayed constraint set need no longer
be delayed because of new information. 
In this case it should be moved
from the delayed constraint set to the collected constraint set and the
usual solvability check made. 
Note that, in general, the notion of which expressions are
``too complicated'' is dependent on the implementation.
In \CLPR\, only the nonlinear constraints are delayed.

Now let us consider an example where the collected constraint set is 
initially empty; then suppose we obtain the constraint

\stuff{
\> V = I * R.
}

\noindent
This is placed in the delayed constraint set. Continuing, if the
next constraint is

\stuff{
\> V = 10
}

\noindent
it may be added to the collected constraint set, but note that it is still
not easy to decide whether the two constraints together are solvable
Now consider what happens if the next constraint is

\stuff{
\> R = 5.
}

\noindent
This gives us enough information to make the delayed constraint linear, so
we simply remove this constraint from the delayed constraint set, 
place it in the collected constraint set,
and check that it is solvable, which of course it is. Note that the delayed
constraint set may have contained other constraints, which may have to
remain there until much later. Also note that because of this delay
mechanism, we may continue through a certain computation sequence even
though the collected and delayed constraint sets together are not solvable.
In the worst case it can result in an infinite loop.
This is the price we pay for an efficient decision algorithm.

As we have already stated, in the \CLPR\ system
a linear equation or inequality is always considered to be 
sufficiently simple to be solved immediately, but nonlinear
constraints are delayed until they become linear.
\chgbarbegin
This includes the functions {\tt sin/1}, {\tt arcsin/1}, {\tt cos/1}, 
{\tt arccos/1}, {\tt pow/2},
{\tt max/2}, {\tt min/2} and {\tt abs/1} 
\index{{\tt sin/1}}\index{{\tt arcsin/1}}\index{{\tt cos/1}}
\index{{\tt arccos/1}}\index{{\tt pow/2}}
\index{{\tt max/2}}\index{{\tt min/2}}\index{{\tt abs/1}}
which are delayed until they become 
simple evaluations in one direction or another. 
\chgbarend
This means that {\tt sin} and  
{\tt cos} require the input to be ground, 
while {\tt pow} requires at least two
out of three arguments to be ground, except in cases such as 

\stuff{
\> X = pow(Y, Z)
}

\noindent
where Z = 0.
The reason is that $Y^0$ is defined to be $1$ for all values of Y. 
Note that while this
is sufficient to determine the value of X, Y remains non-ground. There are
similar special cases when Z is $1$, and when Y is $0$ or $1$.
\chgbarbegin
The functions {\tt arcsin} and {\tt arccos} are delayed until either the
input is ground or the result of the function is ground. They are also
different in that they are functions and the input domain for {\tt arcsin}
ranges from $-\pi/2$ to $\pi/2$ and {\tt arccos} from $0$ to $\pi$
whereas {\tt sin} and {\tt cos} are defined for any number in radians.
Thus {\tt sin} and {\tt cos} behave as relations which is non-invertible
while {\tt arcsin} and {\tt arccos} are true functions which are invertible.
See Section \ref{nl-section} for a more precise definition
of the delaying conditions
for the different nonlinear functions.
\chgbarend

As a final example, consider the {\tt mortgage} program
in Chapter \ref{syntax-examples}, and consider the goal:

\stuff{
\> ?-  mortgage(120, 2, IR, 0, 80).
}

\noindent
This will give rise to nonlinear constraints,
and the system returns a quadratic equation as the answer constraint:

\stuff{
\> 80 = (0.1*IR + 40) * (0.000833333*IR + 1)
}

\noindent
and indicates that this is an unsolved answer.
Note that while \CLPR\ cannot
determine whether this equation is solvable, the equation indeed
describes the correct answer.

\section{The \CLPR\ Operational Model} \index{operational model}

We now precisely but informally define the operational model of \CLPR.
A goal {\tt G} is written in the form {\tt C, D ?- E} where 
{\tt C} is a satisfiable collection of constraints, 
{\tt D} a collection of nonlinear constraints
called the {\em delayed constraints}, and \index{delayed constraint}
{\tt E} a sequence of atoms and constraints.  
In what follows, we define how such a goal
is reduced into another in the context of an ongoing derivation.

In reducing a goal {\tt C, D ?- E}, 
\CLPR\ either selects an element from {\tt E},
call this a {\em forward} reduction, or selects a
constraint from {\tt D}, call this
a {\em wakeup} reduction.  Initially,
{\tt C} and {\tt D} are empty, and \CLPR\ attempts to
make a forward reduction.

\noindent
{\bf Forward reductions}

If {\tt E} is empty, then we say that the
goal is {\em terminal}, and no more reduction of the goal is possible.
If {\tt D} is also empty, then the derivation is {\em successful};
otherwise, the derivation is {\em conditionally successful} (depending
on the nonlinear constraints).

Now consider the case where {\tt E} is nonempty;
let {\tt E0} denote the first element of {\tt E} and
let {\tt E2} denote the remaining subsequence of {\tt E}.

If {\tt E0} is an atom, then {\tt E0} will be selected for atom reduction
in the manner described above.
First, an appropriate program rule will be selected. 
The atom and rule head will then be matched, giving rise to
a collection of constraints, which we will write as {\tt M1 \& M2}
where {\tt M1} consists only of linear constraints and {\tt M2}
only of nonlinear ones.  The new goal consists of 
(a) {\tt C \& M1} in its first component; 
(b) {\tt D \& M2} in its second component, and
(c) the body of the rule and {\tt E2}, sequenced in this order,
in its third component.

If {\tt E0} is a linear constraint, then the reduced goal
is {\tt C \& E0, D ?- E2} providing {\tt C \& E0} is satisfiable;
otherwise there is no reduced goal and the derivation is 
{\em finitely failed}.

Finally, if {\tt E0} is a nonlinear constraint, then the reduced
goal is {\tt C, D \& E0 ?- E2}.  That is, the constraint {\tt E0} is simply
delayed.

\noindent
{\bf Wakeup reductions}

Let the goal at hand be {\tt C, D ?- E}.  This reduction step starts by
considering whether there is a delayed constraint {\tt D0} in {\tt D}
which is in fact linear. That is, {\tt C} implies that {\tt D0}
is equivalent to a linear constraint.  If there is no such delayed
constraint, then no reduction is performed.

Otherwise, consider the case in which {\tt C} is inconsistent 
with this linear constraint.  Here reduction is not possible and
a finitely failed derivation is obtained.  However, if {\tt C} is consistent 
with the linear constraint, then the reduced goal is 
{\tt C \& D0, D2 ?- E} where {\tt D2} is result of deleting
{\tt D0} from {\tt D}.

\section{Meta--programming}
\label{meta-section}
\index{meta-programming}

In the context of Prolog, meta--programming refers to the destruction and
construction of rules and terms, and the examination and modification of the
rulebase. All of the same issues arise in \CLPR. However,
some extra facilities are needed because of the special nature
of arithmetic terms and constraints.
Furthermore, some of the remaining ones must be modified. 
For example, without such extra facilities and modifications,
there is no way that a \CLPR\ program can distinguish the two terms 
{\tt p(3 - 1)} and {\tt p(1 + 1)} since they are semantically identical. 

More specifically, the extra facilities and modifications are needed to:
\begin{itemize}
\item 
	make arithmetic terms be interpreted syntactically, by introducing a
	coded form;
\item 
	convert coded forms of arithmetic terms into the 
	appropriate arithmetic terms;
\item 
	obtain a coded form of [some projection of] the current constraint set; 
\item 
	add appropriate constraints to asserted rules;
\item 
	examine the rulebase completely syntactically.
\end{itemize}

\subsection{{\tt quote/1} and {\tt eval/1}}
\index{{\tt quote/1}}
\index{{\tt eval/1}}

First we introduce the macro-like operator 
{\tt quote/1}\index{quote}. 
This is expanded in an outer-most first
fashion when expressions are first read. 
The argument of the {\tt quote} operator is translated to a version in 
which all arithmetic operators are translated to a special coded form, 
which is not otherwise directly accessible to the programmer. 
This coded form can then be treated like a functor term.
In this discussion, such coded forms of arithmetic function 
symbols will be be represented with a caret over them. 
For example, the rule

\stuff{
\> p(X, Y, quote(X + Y)).
}

\noindent
would be read in as

\stuff{
\> p(X, Y, X \unintt{+} Y).
}

\noindent
and so on. Furthermore, the {\tt quote} operator passes through all other
function symbols, constants, variables etc. without changing them. Thus for
example, the rule

\stuff{
\> q(X,Y) :- X = quote(f(g(Y), 2 * Y)).
}

\noindent
becomes

\stuff{
\> q(X,Y) :- X = f(g(Y), 2 \unintt{*} Y).
}

\noindent
Of course, the original form of the rule is always shown when listing the
database, etc., but when printing a term, coded function symbols  
are printed preceded by a caret\footnote{
In this manual, we take the liberty of placing the caret as
an accent for readability}. 
For example,
the query {\tt ?- q(X, 5).} to the above rule would yield the answer 
{\tt X = f(g(5), 2 \^{}* 5)}. 
Note that that the caret form of coded terms cannot be input directly,
but only through the use of {\tt quote}. 
Additionally, to facilitate manipulating programs which themselves use
meta-programming facilities, we need coded forms of the {\tt quote} operator
itself, as well as the new {\tt eval} interpreted function symbol, 
which will be described below. 
This is why {\tt quote} is expanded outer-most first. For example, 

\stuff{
\> P = quote(p(quote(X + Y), X + Y)) {\rm \ expands to} \\
\>\>	P = p(\unintt{quote}(X\unintt{+}Y), X\unintt{+}Y)).
}

\noindent
Thus an occurrence of
{\tt quote} that appears within the scope of another {\tt quote} will be
translated to \unintt{quote}, and will not be quote-expanded.
The {\tt eval} interpreted function can be coded by using {\tt quote}
as well, for example,

\stuff{
\> X = quote(eval(1 + 2)) {\rm gives} \\
\>\> X = \unintt{eval}(1 \unintt{+} 2).
}

Now, the major linguistic feature for meta--programming with constraints 
is the interpreted function symbol {\tt eval}\index{eval}
which converts a coded term to the term it codes.  It passes through
uninterpreted function symbols, other than those that are coded forms of
interpreted ones, without changing them.  Likewise for constants and
interpreted function symbols.  Some examples:

\stuff{
\> X = 1 \unintt{+} 2, U = eval(X) {\rm \ implies} \\
\>\>	U = 3. \\
\> X = Y \unintt{+} Z, U = eval(X) {\rm \ implies} \\
\>\>	U = eval(Y) + eval(Z). \\
\> X = Y \unintt{+} Z, U = eval(X), Y = 1, Z = 2 {\rm \ implies} \\
\>\>	U = 3.
}

\noindent
The function {\tt eval} has no effect on uninterpreted functors.
For example, the goal

\stuff{
\> ?- X = f(a, g(c)), U = eval(X).
}

\noindent
results in both {\tt U} and {\tt X} being {\tt f(a, g(c))}.  However,

\stuff{
\> ?- X = f(Y, g(c)), U = eval(X).
}

\noindent
results in {\tt U} being {\tt f(eval(Y), g(c))}, 
as the ``best'' representation of terms
containing {\tt eval} is that with {\tt eval} pushed inwards as far as
possible.  

Formally, the meaning of {\tt quote} and {\tt eval} are given
by the axioms: \index{quote}\index{eval}

\begin{center}
$
\begin{array}{lcl}
{\tt eval}(\unint{f}(t_1, \cdots, t_n)) & = & f({\tt eval}(t_1), \cdots,
{\tt eval}(t_n)), \ n \geq 0 \\
{\tt eval}(g(t_1, \cdots, t_n)) & = & g({\tt eval}(t_1), \cdots,
{\tt eval}(t_n)), \ n \geq 0  \\
{\tt eval}(\unintt{quote}(t)) & = & t
\end{array}
$
\end{center}

\noindent
where $f$ ranges over all arithmetic function symbols, $g$ ranges over
all uncoded function symbols different from {\tt eval}, and $t$, $t_1,
\cdots , t_n$ range over terms.

In general, deciding the satisfiability of constraints
involving {\tt quote} and {\tt eval} is a nontrivial problem.
Consider for example the two equations: 

\[
\begin{array}{rcl}
f(eval^2(x), eval^2(y)) & = &
f(\unintt{quote}(eval^4(y)), \unintt{quote}(eval^3(x))) \\
f(eval^3(x), eval^4(y)) & = &
f(\unintt{quote}(eval^2(y)), \unintt{quote}(eval^2(x)))
\end{array}
\]

\noindent
The first of these constraints is solvable, while the second is not.
There is in fact an algorithm to deal with such constraints in
their full generality.  However, for efficiency reasons,
\CLPR\ implements a partial algorithm: maintaining constraints 
so that {\tt eval} appears only in the form {\tt X = eval(Y)},
{\em
these equations are delayed until the argument of {\tt eval} is constructed.
}
In fact, the delay of such {\tt eval} equations is implemented
in much the same way as nonlinear equations.

For example, consider the goal

\stuff{
\> ?- X = quote(U + 1), eval(X) = 5, Y = eval(U) - 5.
}

\noindent
After the first constraint, {\tt X} is equal to {\tt U \unintt{+} 1}, 
but after the second constraint, {\tt eval} goes as far through {\tt X} 
as it can, so we obtain the simplified constraint {\tt eval(U) + 1 = 5},
which is further simplified to {\tt eval(U) = 4}.  Hence the third 
constraint results in {\tt Y} being {\tt -1}. 

However, if the goal were permuted to

\stuff{
\> ?- eval(X) = 5, Y = eval(U) - 5, X = quote(U + 1).
}

\noindent
the first and second constraints both result in delayed {\tt eval} 
constraints. The third constraint wakes the first delayed 
{\tt eval} since {\tt X} is now 
constructed, resulting in the constraint {\tt eval(U) + 1 = 5} again, 
which, together with the second delayed {\tt eval} constraint --- which 
is not awakened --- results in {\tt Y} being grounded to {\tt -1} again.

As a final example, consider the goal

\stuff{
\> ?- eval(X) + eval(Y) = 4, eval(X) - eval(Y) = 1.
}

\noindent
which is rather silly in isolation, but could arise as the result of a longer
computation. In this case, the answer constraints are 
{\tt eval(X) = 2.5, eval(Y) = 1.5} 
although the values of {\tt X} and {\tt Y} cannot be determined uniquely. 
For example, {\tt X} might be {\tt 2.5}, or {\tt 1 \unintt{+} 1.5}, etc.
It should be noted that the {\tt eval} mechanism described here is an 
approximation to that proposed in \cite{HEINTZE89}.

\subsection{{\tt rule/2, retract/1} and {\tt assert/1}}
\label{assert-section}
\index{{\tt rule/2}}
\index{{\tt retract/1}}
\index{{\tt assert/1}}

Next we consider how these basic facilities may be used for reasoning
about programs (see also Section \ref{dynamic-section} which describes
how to use the dynamic code facilities).  
The canonical application for such reasoning is
the meta-circular interpreter, discussed in detail in 
\cite{HEINTZE89}.
Like the {\tt clause/2} predicate of Prolog, \index{clause}
we require a system predicate 
{\tt rule/2} such that the goal {\tt ?- rule(H, B)} behaves as if there 
were facts {\tt rule(E, F)} for each rule {\tt E :- F} in the program 
(and of course {\tt rule(A, true)} for each fact {\tt A}). 

There is, however, one aspect of {\tt rule} which has no analog in
{\tt clause}: arithmetic function symbols will become coded.
More precisely, the system predicate {\tt rule} 
behaves as if there were facts {\tt rule(quote(E), quote(F))}
for each rule {\tt E :- F} in the rulebase 
(and {\tt rule(quote(A), true)} for each fact {\tt A}).
We note that a direct analog to {\tt clause} can be written in terms of 
{\tt rule}: \index{clause}\index{analog to clause}

\stuff{
\> analog\_to\_clause(H, B) :- \\
\>\>functor(H, Name, Arity), \\
\>\>functor(H1, Name, Arity), \% rule needs a constructed head \\
\>\>eval(H) = eval(H1), \\
\>\>rule(H1, eval(B)).
}

\noindent
In a similar fashion, the \CLPR\ system predicate {\tt retract/1}
is like that in PROLOG but differs in that one matches arithmetic
function symbols with their coded forms.
As before, a direct analog to the PROLOG's {\tt retract} can be written
as follows:\index{analog to retract}

\stuff{
\> analog\_to\_retract(eval(R)) :- \\
\>\>functor(R, Name, Arity), \\
\>\>functor(R1, Name, Arity), \% retract needs a constructed argument \\
\>\>eval(R) = eval(R1), \\
\>\>retract(R1).
}

\noindent
Now consider the following example program:

\stuff{
\> {\rm (a)} \hspace*{5mm} p(1, 1.5). \\
\> {\rm (b)} \hspace*{5mm} p(X, Y) :- Y = 2 * X. \\
\> {\rm (c)} \hspace*{5mm} p(X, 2 * X). \\
\> {\rm (d)} \hspace*{5mm} p(X, 2 + X). 
}

\noindent
The goal {\tt ?- retract(quote(p(X, 2*X)))} removes only the rule (c).
The goal 

\stuff{
\> ?- analog\_to\_retract(p(X, 2*X))
}

\noindent
on the other hand, should remove rules (c) and (d).

As explained in  \cite{HEINTZE89}, {\tt assert/1} in \CLPR\
differs from that in PROLOG not just because of term codings;
additional constraints may have to be added to the asserted rule.
For example,

\stuff{
\> ?- X + Y > 2, assert(p(X, Y)).
}

\noindent
results in the rule

\stuff{
\> p(X, Y) :- X + Y > 2.
}

\noindent
As another example, the goal:

\stuff{
\> ?- \> X + Y = 2, X >= 0, Y - 2*X <= 2, X > W, Y - X >= 1, \\
\>\>	assert(p(X, Y)).
}

\noindent
asserts the rule:

\stuff{
\> p(X, Y) :- Y = -X + 2, X <= 0.5, -X <= 0.
}

\noindent
Note that a considerable simplification of the initial constraints has
occurred.  More generally, this supports a technique of constraint partial
evaluation.  This technique consists of
executing a query, and then using the simplified form of the answer
constraints to construct new rules.  These new rules represent a
specialization of the program with respect to that query.
For example:

\stuff{
\> resistor(V, I, R) :- V = I * R. \\
\> ?- \> resistor(V, I1, R1), resistor(V, I2, R2), \\
\>\>	I = I1 + I2, \\
\>\>	assert( parallel\_resistors(V, I, R1, R2)). 
}

\noindent
results in the assertion of a rule describing the 
equivalent voltage-current relationship
of a pair of resistors connected in parallel\footnote{
The actual names of variables in the rule being asserted will be internally
constructed names but we will use the original ones for clarity}:

\stuff{
\> parallel\_resistors(V, I, R1, R2) :- \\
\>\>    V = I2 * R2, \\
\>\>	V = (I - I2) * R1.
}

\noindent
The facilities we have discussed for adding rules to the database have
provided no control over the exact syntax of the rule added.  For example
constraints may be simplified and/or rearranged before the rule is added.
It is particularly important in some applications to have complete control
over the syntax of rules added to the database.
This control is provided by using a coded form of
the rule to be asserted, where \p{assert} of a
coded rule is defined to add the rule that is coded.  For
example, the goal

\stuff{
\> ?- assert(quote( p(X, X + X) :- X - 3 > 0 )).
}

\noindent
asserts the rule

\stuff{
\> p(X, X + X) :- X - 3 > 0.
}

\noindent
In contrast, the goal 

\stuff{
\> ?- assert(p(X, X + X) :- X - 3 > 0).
}

\noindent
could, for example, add the (semantically equivalent) rule:

\stuff{
\> p(X, Y) :- Y = 2*X, Z = X - 3, Z > 0.
}

\section{Output} \index{output} \label{dump-section} \index{dump}

An important feature of the \CLPR\ system is its ability
to output the collected constraints of a successful derivation in 
a simpler form.  In a typical derivation, thousands of constraints
may be collected, and printing them out without simplification would
lead to an unusable answer.  When a derivation succeeds the output
module of \CLPR\ is invoked to print the constraints relating the
variables in the goal.  The module can also be invoked using the
system predicate {\tt dump([X,Y,...,Z])}, discussed later.

The \CLPR\ system attempts to simplify \index{projection}
the constraints in two ways: by projecting the constraints onto 
a set of {\em target\/} variables \index{target variables}
(those appearing in the original goal or given by the 
user in the argument of {\tt dump}), and by eliminating
redundancy in the constraints.  Ideally the output constraints will
only involve target variables and be free of redundancy, but
this will not always be possible.

Recall that there are constraints of four different forms:
\begin{itemize}
\item
functor constraints, e.g. {\tt X = f(Y, a, g(Y))}
\item
linear equations, e.g. {\tt 3*X + 4*Y = 6}
\item
linear inequalities, e.g. {\tt 3*X > 4 + Y}
\item
non-linear equations, 
e.g. {\tt X = Y * Z, T = pow(U, V), U = eval(V)}\footnote{
	Delayed constraints involving {\tt eval} are treated 
	like nonlinears.
}
\end{itemize}
Each of these constraint types is handled differently and in turn.

\subsection{Outline of Algorithm}

In this section, we outline how the output is obtained to give a flavor
of the kinds of simplifications and reductions that are possible
in the answer constraints.

Functor equations are handled first, and in much the same way as in
PROLOG. The constraints are stored in solved form using bindings, and
printing the simplest form of each target variable simply
involves printing their term representation. 
For example

\stuff{
\> ?- X = f(Y, Z), Z = g(a, Y), dump([X, Y]).
}

\noindent
results in the output

\stuff{
\> X = f(Y, g(a, Y)).
}

\noindent
Note that there is no equation for {\tt Y} since it is its own 
term representation.   With functor equations, it is not always
possible to present the output in terms of target variables alone,
and some non-target variables are printed out using an internal name.
For example,

\stuff{
\> ?- X = f(Y, Z), Z = g(a, Y), dump([X]).
}

\noindent
results in an output such as

\stuff{
\> X = f(\_h6, g(a, \_h6)).
}

\noindent
Linear equations are used to substitute out non-target variables
in the following manner.  If $E$ is a linear
equation containing non-target variable $X$, then we rewrite $E$
into the form $X = t$ and substitute $t$ for $X$ in all the other
constraints (including functor equations, inequalities and non-linear
equations).  Consider, for example

\stuff{
\> ?- T = 3 + Y, X = 2 * Y + U, Z = 3 * U + Y, dump([X, T, Z]).
}

\noindent 
First, we eliminate $Y$ using the first equation $Y = 3 - T$ 
and obtain

\stuff{
\> X = 2 * T - 6 + U, Z = 3 * U + T - 3.
}

\noindent
Then we eliminate $U$ using the the first equation and obtain

\stuff{
\> Z = 3*X - 5*T + 15.
}

\noindent
This is the final answer since only the variables {\tt X, T} and {\tt Z}
remain.
\ignore{
This kind of elimination proceeds, and eventually the final equation is used
to eliminate $T$.  This is the correct answer since there is no
constraint directly between $X$ and $Z$, or more formally, given any
value pairs for $X$ and $Z$ the above constraints are satisfiable.
}

Linear inequalities are more difficult to handle than linear
equations.
We will not go into the details of how variables can be
eliminated from inequalities except to mention that
a variation of Fourier-Motzkin elimination \cite{SCHRIJVER86}
with some improvements is used (see \cite{JAFFAR92B} for more details).
In general, eliminating variables from inequalities can be
expensive and the projection can contain an exponential number of inequalities.
\ignore{
  First, they do not allow us to eliminate variables from
other parts of the constraint set.  
Second, while in the constraints handled above redundancy is eliminated
automatically by the solver, redundancy of linear inequalities is
present and expensive to remove. Furthermore
the variable elimination method we use
for linear inequalities, Fourier-Motzkin elimination \cite{SCHRIJVER86},
is expensive and can cause exponential explosions in the number of 
redundant constraints if not handled carefully.  

The Fourier-Motzkin elimination method 
\index{Fourier-Motzkin algorithm}
eliminates a variable $X$ from
some inequalities using the following approach.  Separate the
inequalities into three sets: inequalities without $X$, inequalities
that can be written $X > t_i$ or $X \geq t^=_i$, and inequalities
that can be written $s_j > X$ or $s^=_j \geq X$.
Then we obtain the new constraints $s_j > t_i$, $s_j > t^=_i$,
$s^=_j > t_i$ and $s^=_j \geq t^=_i$, and remove all the constraints
containing the variable $X$.  In the worst case this transformation
can square the number of constraints, and it usually produces
considerable redundancy.  If we do not eliminate any redundancy the
Fourier-Motzkin elimination can can quickly explode the number of 
inequalities.  For example, starting from $32$ constraints we can generate
in $4$ steps a total of $390,417,582,083,242$ constraints; of these,
only $2$ constraints being non-redundant.

Full redundancy elimination on sets of linear inequalities is a non-trivial
task and quite computationally expensive.  
A compromise solution was described by Kohler \cite{KOHLER67};
he devised an inexpensive test that detects many redundant 
constraints produced by Fourier-Motzkin elimination, which 
allows us to detect their
redundancy {\em before\/} they are actually produced.  
\index{Fourier algorithm}
This
simple check allows us to perform enough redundancy elimination
to avoid the worst case behavior of  Fourier-Motzkin and at very
little computational cost.  In the \CLPR\ system, we use an adaptation
of the Kohler method which incorporates more redundancy removal
\cite{JAFFAR92B}.
}

We finally deal with the nonlinear equations.
In general, the algorithm here simply outputs each
nonlinear equation unless it has been used as a substitution.
We will not define formally what exactly constitutes a substitution,
but will discuss some examples.
Recall that each non-linear constraint takes the form
$X = Y * Z, ~ X = sin(Y), ~ X = cos(Y), ~
X = pow(Y, Z), ~ X = max(Y,Z), ~
X = min(Y, Z)$ or $X = abs(Y)$.  Each of these equations can be used
to substitute for $X$ if $X$ is a non-target variable. 
For example,

\stuff{
\> ?- Y = sin(X), Y = cos(Z), dump([X,Z]).
}

\noindent
leads to the output

\stuff{
\> sin(X) = cos(Z).
}

\noindent
As in the case for functor equations, we cannot in practice
eliminate all non-target variables appearing in non-linear constraints. 
As before, we display any non-target variable using an internal name.  

\noindent
{\bf A Complete Example}

Consider the goal

\stuff{
\>?- \> X = f(V, M), V = a, N = 2 * T, Y = 4 * T, Z = R + T, M = N * R, \\
\>\>	Y + Z >= U, U > T, U >= R + N, \\
\>\>	dump([X, Y, Z]).
}

\noindent
First we eliminate $V$ by substitution obtaining

\stuff{
\> X = f(a, M), N = 2 * T, Y = 4 * T, Z = R + T, M = N * R, \\
\> Y + Z >= U, U > T, U >= R + N
}

\noindent
Next we eliminate $N$ using the second constraint obtaining

\stuff{
\> X = f(a, M), Y = 4 * T, Z = R + T, M = (2 * T) * R, \\
\> Y + Z >= U, U > T, U >= R + 2 * T
}

\noindent
Next we eliminate $T$ using the second constraint obtaining

\stuff{
\> X = f(a, M), Z = R + 0.25 * Y, M = (0.5 * Y) * R, \\
\> Y + Z >= U, U > 0.25 * Y, U >= R + 0.5 * Y
}

\noindent
Next we eliminate $R$ using the second constraint obtaining

\stuff{
\> X = f(a, M), M = (0.5 * Y) * (Z - 0.25 * Y), \\
\> Y + Z >= U, U > 0.25 * Y, U >= Z + 0.25 * Y
}

\noindent
Next we eliminate $U$ from the inequalities
(and here the individual steps taken may not
be so obvious), obtaining

\stuff{
\> X = f(a, M), M = (0.5 * Y) * (Z - 0.25 * Y), \\
\> 0.75 * Y + Z > 0, 0.75 * Y >= 0
}

\noindent
Finally, we eliminate $M$ using the second constraint, and as output
we obtain (after performing some straightforward scaling) the constraints

\stuff{
\> X = f(a, (0.5 * Y) * (Z - 0.25 * Y)), \\
\> 0 < Z + 0.75 * Y, \\
\> 0 <= Y
}

\noindent
We finally remark that we can obtain an empty
output using the algorithm just outlined.  This indicates
that there are no restrictions on the values that the target
variables can take.  For example,

\stuff{
\> ?- T = 3 + Y, X = 2 * Y + U, Z = 3 * U + Y, dump([X, Z]).
}

\noindent
results in no constraints at all.  In such cases, 
the distinguished predicate {\tt real/1}
is then used to indicate that certain variables are arithmetic,
and that no further constraints are upon them.
In this example, we will output the constraints

\stuff{
\> real(X), real(Z).
}

\subsection{The {\tt dump} System Predicates}
\label{dump-preds}
\index{{\tt dump/1}}
\index{{\tt dump/2}}
\index{{\tt dump/3}}
\index{dump}

The basic facility for output in \CLPR{} is the system
predicate {\tt dump/1}, mentioned above, whose argument is a list of
target variables. \index{target variables} 
Note that, to use this predicate, the
target variables must appear explicitly in the argument
(as in {\tt dump([A, B])}) and not be passed in
(as in {\tt X = [A, B], dump(X)}).  
This is because the names of the target variables
are actually used in the output. 
The ordering of variables in the list is used to specify a priority
on the variables with the later variables having a higher priority.
Since {\tt dump} outputs constraints, there are many equivalent
forms of the same set of constraints and the priority ordering is
used to express higher priority variables in terms of the lower ones.
This gives one form of control over the output from {\tt dump}.
For example, the goal

\stuff{
\>?- X = 2 * Y + 4, dump([X, Y]) \\
\>\> {\rm gives} Y = 0.5 * X - 2 
}

\noindent
whereas the reverse order would give back the original constraint.


The predicate {\tt dump/2} is a refinement of {\tt dump/1},
and is designed to be far more flexible.
Its first argument is, as before, a list of target variables.
Its second argument is a list of constants to be used in place of
the original target variables in the output.  For example,

\stuff{
\> ?- Names = [a, b], Targets = [X, Y], X > Y, dump(Targets, Names).
}

\noindent
results in the output {\tt a > b}.  This predicate is useful
when the names of target variables are known only at runtime. 
More precisely, the operation of {\tt dump/2} is as follows:
let the first and second arguments be the lists 
$[t_1, \cdots , t_n]$ and $[u_1, \cdots , u_n]$,
where the $t_i$ and $u_i$ are {\em arbitrary} terms.
Construct new variables $T_1, \cdots , T_n$, and add to the current
collection of constraints the equations 
$T_1 = t_1, \cdots , T_n = t_n$.
Now obtain a projection of the augmented constraints \wrt
$T_1, \cdots , T_n$.
Finally, output this projection renaming each target variable $T_i$ 
by its new name $u_i$.

In meta-programming it can be useful to obtain the coded form of 
the constraints with respect to given target variables. 
This facility is provided by the system predicate {\tt dump/3}.
There are three arguments because it is not sufficient to simply
provide the variables to be projected upon (1st argument) and the variable
that receives the coded form (3rd argument). The 2nd argument is a list of
terms that are to replace the original variables in the coded form, and hence
the lengths of the two lists must be the same. 
For example,

\stuff{
\> ?- \> NewVars = [A, B, C], Targets = [X, Y, Z], X > Y + Z, \\
\>\>	dump(Targets, NewVars, Answer).
}

\noindent
results in the binding \ 
{\tt Answer = [\unintt{-} A \unintt{+} B \unintt{+} C < 0]}.

There are two reasons for having such a second argument.
First, it is very inconvenient to manipulate a coded form containing variables
that have the original arithmetic constraints still imposed on them --- in
particular, printing such a term leads to highly counter-intuitive results.
Second, in many cases it is more convenient to manipulate ground
representations of the coded forms. That is, with syntactic constants
replacing the variables. The terms resulting from manipulation can then have
the original (or other) variables substituted into place easily.

We conclude with a larger example. 
We will assume that the predicate {\tt p/2} sets
up a constraint such that the first argument is a (polynomial) function of 
the second, and that {\tt diff/2} implements symbolic differentiation on 
coded forms of arithmetic constraints. Then, to find the turning point of 
the functional relationship established by {\tt p/2}, we can use the 
following goal:


\stuff{
\> solve(DYDX,X) :- eval(DYDX) = 0. \\
\> p(Y, X) :- \\
\>\>	T = X + 1, \\
\>\>	Y = T * T. \\
\> ?- \> p(Y, X), 
			\>\> \pct {\em \ collect a function Y(X)} \\
\>\>	dump([Y, X], [V, U], Z), 
			\>\> \pct {\em \ get coded form of Y(X)} \\
\>\>	Z = [C], C =.. ['=', V, RHS], 
			\>\> \pct {\em \ assume Z of the form [V = f(U)]} \\
\>\>	diff(RHS, DVDU), 
			\>\> \pct {\em \ symbolic differentiation} \\
\>\>	solve(DVDU, U), 
			\>\> \pct {\em \ find extremum} \\
\>\>	printf("Turning point: X = {\pct}, Y = {\pct}$\backslash$n", [U, V]).
}

\section{Some Programming Techniques}

Here we collect a number of small programs that serve to illustrate
some interesting programming techniques.

\noindent
{\bf A Crypto-arithmetic Puzzle}

Consider one of the
standard crypto-arithmetic puzzles. We require an injective
assignment of digits $0, 1, \cdots , 9$ to
the letters S, E, N, D, M, O, R, Y such that the equation

\begin{verbatim}
          S E N D
        + M O R E
        ---------
        M O N E Y
\end{verbatim}

\noindent
holds. The program first imposes certain constraints on the values. Then
it tries to assign possible values to the letters. The problem is
combinatorially explosive and so a naive generate and test solution 
would be very inefficient.  In contrast, the straightforward
program below runs quickly in \CLPR.

The program illustrates how \CLPR{} can be used to advantage in solving
problems over integer domains.  Because the unsolvability of constraints
in $\cal R$ implies their unsolvability over the integers, \CLPR{} can 
prune the search space significantly without the expense of invoking an
integer solver.  For CLP programs in general, the key issue
is the trade-off between the power and the speed of the constraint-solver:
powerful solvers entail smaller search spaces but are
costlier to run.  For \CLPR{} in particular, the use of
a real-number-based solver to approximate constraint-solving
over a discrete or finite domain is one important realization
of this trade-off.


\begin{verbatim}
solve([S, E, N, D, M, O, R, Y]) :-
        constraints([S, E, N, D, M, O, R, Y]),
        gen_diff_digits([S, E, N, D, M, O, R, Y]).

constraints([S, E, N, D, M, O, R, Y]) :- 
        S >= 0, E >= 0, N >= 0, D >= 0, M >= 0, O >= 0, R >= 0, Y >= 0,
        S <= 9, E <= 9, N <= 9, D <= 9, M <= 9, O <= 9, R <= 9, Y <= 9,
        S >= 1, M >= 1,
        C1 >= 0, C2 >= 0, C3 >= 0, C4 >= 0,
        C1 <= 1, C2 <= 1, C3 <= 1, C4 <= 1,
        M = C1,
        C2 + S + M = O + C1 * 10,
        C3 + E + O = N + 10 * C2,
        C4 + N + R = E + 10 * C3,
        D + E = Y + 10*C4,
        bit(C1), bit(C2), bit(C3), bit(C4).

bit(0).
bit(1).

gen_diff_digits(L) :- 
        gen_diff_digits(L, [0, 1, 2, 3, 4, 5, 6, 7, 8, 9]).
gen_diff_digits([], _).
gen_diff_digits([H | T], L) :- 
        select(H, L, L2), gen_diff_digits(T, L2).
select(H, [H | T], T).
select(H, [H2 | T], [H2 | T2]) :- 
        select(H, T, T2). 

?- solve(S, E, N, D, M, O, R, Y).
\end{verbatim}

\noindent
{\bf Critical Path Analysis}

This program uses local propagation to compute start, completion and float
times for a project network. Significantly, the constraint paradigm allows the
program to compute these values by making only one pass of the project
network, as opposed to the three passes that would be needed using a
conventional programming language.

Most of the program is basically parsing the input and building
an adjacency graph out of the network. Then the latest completion
time and earliest starting time for every node is simply the
minimum of the time required for the outgoing events and maximum of
the time of the incoming events.

% Network is an input project network of the form
%       [ [node1 , node2, time ] .... ]
%       Graph is the critical path graph produced
%       Latest is the latest possible completion time is specified
% cpm/3 is used if the latest time is specified; otherwise use cpm/2

\begin{verbatim}

cpm(Network, Graph, Latest) :-
        build(Network, Graph),
        early_late(Graph, Graph, End, Latest),
        Latest >= End,
        analyse(Graph, Graph).
        
cpm(Network, Graph) :-
        build(Network, Graph),
        early_late(Graph, Graph, End),
        analyse(Graph, Graph).

% Build adjacency graph out of the network ... build([], Graph) :- ...
% Get early start times and latest completion times
% early/4 is used when a ending time is given
% otherwise early/3 assumes that the early start time
% for the end node is equal to the latest completion time

early_late([], _, _, _).
early_late([ad(I, Es, Lc, To, From) | T], G, End, Latest) :-
        setearly(From, To, G, End, Es),
        setlate(To, G, Latest, Lc),
        early_late(T, G, End, Latest).

early_late([], _, _).
early_late([ad(I, Es, Lc, To, From) | T], G, End) :-
        setearly(From, To, G, End, Es),
        setlate(To, G, End, Lc),
        early_late(T, G, End).

setearly([], _, _, _, 0).
setearly([ed(V, C, _, _, _, _) | T],[], G, Es, Es) :-
        !,
        getnode(V, G, Es1, _),
        setmax(T, G, Es1 + C, Es).
setearly([ed(V, C, _, _, _, _) | T], _, G, End, Es) :-
        getnode(V, G, Es1, _),
        setmax(T, G, Es1+C, Es).

setmax([], _, Max, Max).
setmax([ed(V, C, _, _, _, _) | T], G, Max0, Max) :-
        getnode(V, G, Es1, _),
        setmax(T, G, max(Max0, Es1 + C), Max).

setlate([], _, Last, Last).
setlate([ed(V, C, _, _, _, _) | T], G, Last, Lc) :-
        getnode(V, G, _, Lc1),
        setmin(T, G, Lc1-C, Lc).

setmin([], _, Min, Min).
setmin([ed(V, C, _, _, _, _) | T], G, Min0, Min) :-
        getnode(V, G, _, Lc1),
        setmin(T, G, min(Min0, Lc1 - C), Min).

% Search graph for the early & late times for a node
getnode(I,[ad(I, Es, Lc, _, _) | T], Es, Lc).
getnode(I,[H | T], Es, Lc) :-
        getnode(I, T, Es, Lc).

% Compute the other times:
%               Ls - latest start time
%               Ec - earliest completion time
%               Tf - total float time
%               Ff - free float time

analyse([], G).
analyse([ad(I, Es, Lc, To, _) | T], G) :-
        analyse_times(To, Es, Lc, G),
        analyse(T, G).

analyse_times([], _, _, _).
analyse_times([ed(V, C, Ls, Ec, Tf, Ff) | T], Esi, Lci, G) :-
        getnode(V, G, Esj, Lcj),
        X = Esi + C,
        Ls = Lcj - C,
        Ec = Esi + C,
        Tf = Lcj - X,
        Ff = Esj - X,
        analyse_times(T, Esi, Lci, G).

print_analysis(G) :- ...

\end{verbatim}

A goal might be 

\begin{verbatim}
?-      cpm([
                [n1, n2, 4], [n1, n3, 3], [n1, n4, 4], [n2, n5, 7],
                [n2, n3, 1], [n2, n7, 8], [n3, n5, 4], [n4, n6, 2], 
                [n5, n6, 1], [n5, n7, 3], [n6, n7, 4]], G),
        print_analysis(G).
\end{verbatim}

\noindent
{\bf A Simple Circuit Solver}

The following program performs DC analysis on circuits
containing resistors, voltage sources and diodes.
The circuit analysis is decomposed in a hierarchical fashion.
The individual components are modelled directly by constraints
such as Ohm's law.
Then the components are connected together and the global circuit
constraints on the currents and voltages, as specified by
Kirchoff's laws, are used to define the whole circuit.

\begin{verbatim}
solve_dc(C, L) :-
        solve(C, [], L),
        solve_current(L).

% solve for every circuit component
solve([], L, L).
solve([[Comp, Name, Par, Nodes] | T], In, Out) :-
        connect(Name, Nodes, Volts, Amps, In, Tmp),
        component(Comp, Par, Volts, Amps),
        solve(T, Tmp, Out).

% sum of currents at each node are zero
solve_current([]).              
solve_current([n(N, V, IList) | T]) :-
        kcl(IList, 0),
        solve_current(T).

kcl([], 0).
kcl([(Name, I) | T], X) :-
        kcl(T, I + X).

% connect the arcs which meet at a node
connect(Name, [], [], [], L, L).      
connect(Name, [N | T], [V | VR], [I | IR], In, Out) :-
        add_arc(Name, N, V, I, In, Tmp),
        connecting(Name, T, VR, IR, Tmp, Out).

% create the voltage and currents
add_arc(Name, N, V, I, [], [n(N, V, [(Name, I)])]). 
add_arc(Name, N, V, I, [n(N, V, IList) | T], 
                       [n(N, V, [(Name, I) | IList]) | T]).
add_arc(Name, N, V, I, [X | T], [X | T1]) :- 
        add_arc(Name, N, V, I, T, T1).

component(resistor, R, [V1, V2], [I, -I]) :-
        V1 - V2 = I*R.
component(voltage_source, V, [V, 0], [I, -I]).
component(diode, in914, [V1, V2], [I, -I]) :-
        diode(in914, [V1, V2], [I, -I]).
diode(in914, [V1, V2], [I1, I2]) :-
        V = V1 - V2, V < -100,  DV = V+100,  I1 = 10*DV - 0.1.
diode(in914, [V1, V2], [I1, I2]) :-
        V = V1 - V2, V >= -100, V < 0.6, I1 = 0.001*V.
diode(in914, [V1, V2], [I1, I2]) :-
        V = V1 - V2, V >= 0.6, DV = V - 0.6, I1 = 100*DV - 0.0006.
\end{verbatim}

A sample query which returns the currents and voltages in L

\begin{verbatim}
?-      R1 = 100, R2 = 50, V = 20,
        solve_dc([[voltage_source, v1, V, [n1, ground]], 
                  [resistor, r1, R1, [n1, n2]],
                  [resistor, r2, R2, [n2, ground]], 
                  [diode, d1, in914, [n2, ground]]], L).
\end{verbatim}

\chapter{Using the System} \label{chap:system}

The user interface of compiled \CLPR\ is very much like that of a usual
Edinburgh-style Prolog interpreter. In other words, it is quite possible to
use this system while almost completely ignoring the fact that it is
compiler-based. 
In fact, there is no such thing as an interpreted mode and all code
(static and dynamic) is compiled.
All goals are compiled (quickly) before being executed, and any consulted file
is immediately compiled. The rulebase is available for inspection (except
for protected rules) and
can be modified dynamically as long as the relevant relations have been
declared to be {\em dynamic} as described below. Normally the user will
find that consulted files take a little longer than usual to be read in
(because they are being compiled) and that programs will usually
run much more quickly and use less space than in an interpreter. Symbolic
debugging is still possible, as are all other aspects of interactive
programming. However, the user may also take special advantage of the compiler
by creating {\tt clam} files that contain compiled \CLPR\ code that can be
loaded extremely quickly and do not include the overhead of the
original program text, although this rules out certain operations.
\index{clam files}
In short, the system is intended to get the best of both worlds by combining
the flexibility of an interpreter with the efficiency of a compiler.
The experienced PROLOG user may want to skip directly to 
Section \ref{sample-session} which illustrates many of the features,
syntax and user interface of \CLPR{} using an example session.

{
\Large
\ \\
\noindent
\begin{tabular}{ll}
% \hspace{12mm} \= \kill
Note: & Creation of CLAM files has not yet been implemented. \nl
	& However, compilation in \CLPR\ is relatively quick.
\end{tabular}
}

The first operation \CLPR\ performs
is to load the distinguished library file
{\tt init.clpr}.  
\index{init.clpr}
This file must either be in the current working directory,
or in a directory whose path name is defined via the environment
variable {\tt CLPRLIB}, 
\index{CLPRLIB}\index{Environment variables}
or in a directory whose path name is specified
during installation.  This last alternative is explained 
in Chapter \ref{installation-guide}.

\section{Command Line Arguments}\index{command line arguments}
\label{command-section}
The syntax of a command-line is

{\tt clpr} [{\em options}] [$\mbox{{\em filename}}$]

\noindent
where {\em filename} contains a \CLPR\ program.
The following explains the various {\em options} available:

\begin{description}
\ouritem{-cs <n>}
Specify size of code space\index{code space} (default 128,000).
\ouritem{-hs <n>}
Specify size of heap\index{heap} (default 200,000).
\ouritem{-ls <n>}
Specify size of local stack\index{local stack} (default 100,000).
\ouritem{-ss <n>}
Specify maximum number of solver variables\index{solver variables} 
	(default 128,000).
\ouritem{-ts <n>}
Specify size of trail\index{trail} (default 100,000).
\ouritem{-z <r>}
Set internal notion of zero\index{zero} to this small number.
Numbers between $\pm r$ are taken to be equivalent to zero.
\ouritem{-r <int>}
Specify a random number\index{random number seed} seed.
\end{description}

\section{Filenames}\label{sec:filenames}\index{file names}
\index{suffix}
\chgbarbegin
Filenames consulted or read as an input stream
may have an optional implicit suffix added to the filename.
The default suffix is usually ``.clpr'' (``.clp'' for MS/DOS or OS/2)
depending on the installation.
This may be changed by the use of the environment variable 
CLPRSUFFIX\index{CLPRSUFFIX}\index{Environment variables},
which can be set to a list of suffixes separated by colons,
e.g. ".clpr:.clp".
First, the original filename is tried and if that cannot be read
then a suffix is added in the order specified by the list of suffixes.
(Note that in version 1.1 and earlier of \CLPR{}, only the specified filename
was used without any implicit suffixes, but the behavior here is
compatible).
\chgbarend

\section{Queries}\index{answer constraints}\index{queries}\index{goals}

After the system has been initialized, it will prompt the user for a query.
\chgbarbegin
It will continually accept user goals and solving for them
until the session is terminated with a {\tt halt/0} or if
it encounters the end of file (eg: {\tt $^\wedge$D} on UNIX or 
{\tt $^\wedge$Z} on MSDOS).
This again is similar to the style of most PROLOG systems. 
If the user goal failed then the ``{\tt *** No}'' message is output, otherwise
the query is successful and the
resulting answer constraints (the constraints on variables in the
query) are output.
A successful query will also display a ``{\tt *** Yes}'' message,
but if there are other alternatives to try for the query then 
the ``{\tt *** Retry?}'' message is displayed and the 
user is prompted to either press carriage return or enter ``.'' (or ``n'') 
to accept the
answers, or ``;'' (or ``y'') to cause backtracking.
A different prompt is displayed if delayed (nonlinear) 
remain at the end \index{delayed constraint}\index{nonlinear constraint}
of the execution.
The message ``{\tt *** Maybe}'' replaces ``{\tt *** Yes}''
and ``{\tt *** (Maybe) Retry?}'' replaces ``{\tt *** Retry?}''
to indicate that the satisfiability of the nonlinear constraints remaining
has not been decided by \CLPR.
Execution of a query can be interrupted at any time by using
the interrupt keycode ({\tt $^\wedge$C} usually).\footnote{It is however
not absolutely safe to interrupt at any time,
and occasionally at critical stages
an interrupt may cause the system to be internally inconsistent}
\chgbarend
A buffer of the last 50 goals is kept, and may
be examined by the {\tt history/0} (or {\tt h/0}) predicate. 
An old query may be re-executed by just
entering its history number as a goal (eg: {\tt ?- 5.}).

For every top-level query. There is also an {\em implicit dump} on
\index{implicit dump}
the variables in the goal, i.e. the set of answer constraints
using those variables are printed, with the exception that anonymous
variables and also other variables beginning with an ``{\tt \_}''
are ignored.
No implicit dump is performed for goals embedded in a file.
(Note that the output constraints differs from many PROLOG systems
which display the variable bindings produced from execution.)

\section{Loading/consulting and reconsulting programs}
\chgbarbegin

A \CLPR{} source program can be loaded using the {\tt consult/1}
predicate or the more convenient notation{\tt [{\it $<$list of filenames$>$}]},
e.g. {\tt [myprog, mytest]} at the top level prompt loads those
two files.
Loading a program compiles all the rules in that file, makes the 
new predicates in it available for use and also executes any embedded goals.
Unlike some PROLOG systems where consulted files are interpreted
and compilation is done using a different method, all consulted predicates
in \CLPR{} are compiled (usually fairly quickly). 
Note that filenames may have an implicit suffix added as in 
Section \ref{sec:filenames}.
Filenames which are specified directly should consist entirely of lowercase
characters and any other kind of filename, eg. a pathname,
should be surrounded by single quotes.

Reconsulting a file with {\tt reconsult/1} or the notation
{\tt [`{\it $<$list of filenames$>$}]} will if it encounters previous
definitions, erase them and replace them by the new definitions.
By default, a predicate which is redefined will generate a warning.
This may be turned off by executing the system predicate
{\tt warning(redefine\_off)}.
Some PROLOG systems use an alternative notation {\tt [-{\it filename}]}
but in \CLPR{} this conflicts with unary minus.
Also in some systems, consulting and reconsulting are combined together.
In \CLPR{} consulting a previously consulted file with active definitions
will result in warning messages and redefinitions will be ignored.

The special filename {\tt user} denotes that the file to be consulted
or reconsulted
is read from standard input. This allows direct entry of rules which is handy
for quick modifications from the top query level.
More on the organization of consulted files is contained in 
Section \ref{consult-section}.
\chgbarend

\section{Style Checking and Warnings}\index{style checking}\index{warning}
\label{style-section}
\chgbarbegin
\CLPR{} programs can be optionally checked against
some stylistic conventions, also called style checking.
The purpose of the style checking is to give a warning
that the program may potentially contain some  common ``bugs'' when
the style rules are not followed.
It is important to remember that these are merely warnings
and a program may be perfectly correct otherwise.
There are three different kinds of style checking that can be applied
--- {\tt single\_var}, {\tt discontiguous}, {\tt name\_overload}.\footnote{
The first two options are similar to that in Quintus Prolog.
The last is different.}
The option {\tt all} covers all three styles.
By default, style checking is on and individual style checking
can be turned on (off) with {\tt style\_check/1} ({\tt no\_style\_check/1}),
e.g. {\tt no\_style\_check(all)} turns off
all style checking.

\noindent
The different style conventions are as follows:
\begin{description}
\item{{\tt single\_var} ---}
This warns if a variable is used only once within a rule and may
possibly indicate that a variable has been mispelled. Anonymous
variables ({\tt \_}) and also variables prefixed with an underscore
are ignored. An example error is the rule ``{\tt p(X, Y)}'' gives
the following warning message:

\begin{quotation}
\tt
Warning: Style check, singleton variables, rule 1 of q/2 \\
 +++  X, Y
\end{quotation}

\item{{\tt discontiguous} ---}
This style check assumes that all the different rules defining
a predicate occur contiguously within a file and warns if there
is another intervening rule. Common bugs which
can result when this style check is not followed
can be
mispelling the name of a rule, or substituting a ``.'' 
to end a rule when a ``,'' was meant to continue the rule,
e.g. the program ``{\tt p(X) :- X > 0. q(X). p(0) :- r(X).}''
where there the intent is for a comma to be before {\tt q/1}
gives the following warning message:

\begin{quotation}
\tt
Warning, $<$stdin$>$:1: Style check, p/1 is not contiguous
\end{quotation}

\item{{\tt name\_overload} ---}
This checks whether the same predicate name is defined with
different arities.
While it is not uncommon to have different predicates of different
arities with the same name, it may also be indicative of an incorrect
number of arguments, e.g. the program ``p(0,0). p(1). p(2,2).''
gives the following warning message:

\begin{quotation}
\tt
Warning: rule overloading, same name, different arity: \\
 +++  p/1, p/2
\end{quotation}
\end{description}
(Note that when this option has been disabled and then re-enabled,
then rules which were defined before style checking was enabled
will also generate warnings. The additional warnings can be
disabled by using the special system predicate {\tt \$clear\_style\_check/0}.
{\tt style\_check(all\_reset)} also does this,
clearing all previous warnings and
turns on style checking.)

\noindent
Another kind of warning is given when a rule is defined in more than
one file. The basic unit of compilation is a single file and all
the occurences of rules for a predicate have to be defined within
the same file. The exception is that when a file is being reconsulted,
then the new definitions replace the old ones. The compiler will
simply ignore all additions to an existing previously compiled
predicate and by default a warning is given. See also {\tt warning/1}
to control whether warnings are given.
\chgbarend

\section{Sample Session}\label{sample-session}\index{sample session}

This is a sample session with the \CLPR\ system.
Some extra information is given using comments after the \% character.

\begin{verbatim}

% clpr

CLP(R) Version 1.2
(c) Copyright International Business Machines Corporation
1989 (1991) All Rights Reserved

1 ?- f(X,Y) = f(g(A),B).        % some simple ``unification''

B = Y
X = g(A)

*** Yes

2 ?- X = Y + 4 , Y = Z - 3, Z = 2.      % simple arithmetic evaluation

Z = 2
Y = -1
X = 3

*** Yes

3 ?- X + Y < Z, 3 * X - 4 * Y = 4, 3 * X + 2 * Y = 1.

Y = -0.5
X = 0.666667
0.166667 < Z

*** Yes

4 ?- X + Y < Z, 3 * X - 4 * Y = 4, 2 * X + 3 * Z = 1.

Y = -1.125*Z - 0.625
X = -1.5*Z + 0.5
-0.0344828 < Z

*** Yes

5 ?- history.
1       f(X, Y) = f(g(A), B).
2       X = Y + 4, Y = Z - 3, Z = 2.
3       X + Y < Z, 3 * X - 4 * Y = 4, 3 * X + 2 * Y = 1.
4       X + Y < Z, 3 * X - 4 * Y = 4, 2 * X + 3 * Z = 1.

*** Yes

6 ?- 2.                                  % run second goal again 
X = Y + 4, Y = Z - 3, Z = 2.

Z = 2
Y = -1
X = 3

*** Yes

7 ?- ['examples/fib'].          % consult (load) a program

>>> Sample goal: go/0

*** Yes

8 ?- ls fib.                    % look at the program

fib(0, 1).
fib(1, 1).
fib(N, X1 + X2):-
        N > 1,
        fib(N - 1, X1),
        fib(N - 2, X2).

*** Yes

9 ?- fib(5,F).                  % only one answer to this

F = 8

*** Retry?;

*** No

10 ?- F > 7, F < 9, fib(N,F).  % only ask for the first answer

N = 5
F = 8

*** Retry? 
11 ?- [`'examples/mortgage'].  % use "`" to reconsult

>>> Sample goals: go1/0, go2/0

*** Yes

12 ?- ls.               % look at the entire rulebase

h:-
        history.

fib(0, 1).
fib(1, 1).
fib(N, X1 + X2):-
        N > 1,
        fib(N - 1, X1),
        fib(N - 2, X2).

go:-
        printf(\nFib(14) = , []),
        ztime,
        fib(14, X),
        ctime(T1),
        printf(% (Time = %)\n, [X, T1]),
        printf(Fib-1(610) = , []),
        ztime,
        fib(Y, 610),
        ctime(T2),
        printf(% (Time = %)\n, [Y, T2]).

mg(P, T, I, B, MP):-
        T = 1,
        B = P + P * I - MP.
mg(P, T, I, B, MP):-
        T > 1,
        mg(P * (1 + I) - MP, T - 1, I, B, MP).

go1:-
        ztime,
        mg(999999, 360, 0.01, 0, M),
        ctime(T),
        printf(Time = %, M = %\n, [T, M]).

go2:-
        ztime,
        mg(P, 720, 0.01, B, M),
        ctime(T),
        printf(Time = %\n, [T]),
        dump([P, B, M]).

*** Yes

13 ?- [`'examples/mortgage'].
Warning: mg/5 has been redefined

>>> Sample goals: go1/0, go2/0

*** Yes

14 ?- ls.

h:-
        history.

fib(0, 1).
fib(1, 1).
fib(N, X1 + X2):-
        N > 1,
        fib(N - 1, X1),
        fib(N - 2, X2).

go:-
        printf(\nFib(14) = , []),
        ztime,
        fib(14, X),
        ctime(T1),
        printf(% (Time = %)\n, [X, T1]),
        printf(Fib-1(610) = , []),
        ztime,
        fib(Y, 610),
        ctime(T2),
        printf(% (Time = %)\n, [Y, T2]).

mg(P, T, I, B, MP):-
        T = 1,
        B = P + P * I - MP.
mg(P, T, I, B, MP):-
        T > 1,
        mg(P * (1 + I) - MP, T - 1, I, B, MP).

go1:-
        ztime,
        mg(999999, 360, 0.01, 0, M),
        ctime(T),
        printf(Time = %, M = %\n, [T, M]).

go2:-
        ztime,
        mg(P, 720, 0.01, B, M),
        ctime(T),
        printf(Time = %\n, [T]),
        dump([P, B, M]).

*** Yes

15 ?- go2.
Time = 0.25

M = -7.74367e-06*B + 0.0100077*P

*** Retry?

16 ?- [user].
p(X) :- writeln(X).
^D
*** Yes

17 ?- p(hello).
hello

*** Yes

\end{verbatim}

\section{Organization of Consulted Files}
\label{consult-section}
\index{consulted files}

Slightly more care than usual must be taken in organizing program files in
compiled \CLPR. A file consists of a number of {\em chunks}. Each chunk
consists of a zero or more rules (defined in the usual way) possibly followed
by a goal. That is, a goal always closes off a chunk, and the end of the file
closes off the last chunk if a goal has not done so.
\chgbarbegin
A relation may not span more than one chunk unless it has been declared to be
{\em dynamic} (see below) before the first rule defining it. 
Defining a relation statically in more than one chunk will generate
a warning message stating that the new definitions will be ignored is given.
However if one is reconsulting then the new definitions will replace the
ones defined in the previous chunk. A warning message that the redefinition
has taken place is also given.
However, if such a redefinition during a reconsult is not possible when
the earlier definition has been protected (using the {\tt prot/2} predicate),
in which case a warning is printed and the new
definition is ignored.\index{prot}
\chgbarend
The motivation for this restriction is that the state of the rulebase needs to
be well defined whenever a goal is encountered in the consulted file.

There may be three kinds of goals in any consulted file. \index{goals}
All three kinds are considered to be identical (and behave in the usual way) 
when they are encountered in a source file that is being consulted. 
However, they are different when a source file is first compiled 
and when the {\tt .clam} file is consulted. 
All goals of the form {\tt :- goal}
are only executed during the compilation stage. 
Those of the form {\tt ::- goal}
are only executed during the consultation of the compiled code, and the goals
of the traditional form {\tt ?- goal} are executed twice: once during
compilation and once during consultation. 
In summary:

\stuff{
\> :- goal. \\
\>\>	{\em is executed during compilation of the source file.} \\
\> ::- goal. \\
\>\>	{\em is executed during consultation of the {\tt .clam} file.} \\
\> ?- goal. \\
\>\>	{\em is executed during compilation \underline{and} at runtime.}
}

\noindent
The first kind of goal might be used
for compiler directives and messages to whoever is watching while some code is
being compiled. The second kind might be used for making a program run itself
straight after it is loaded. Finally, the third kind of goal is useful for
things like operator declarations which need to be present for the remainder
of a program to parse correctly and also when the program is running so that
terms will print correctly, etc.
\chgbarbegin
An embedded goal that fails during execution will generate
a warning message (see also {\tt warning/1}).
\chgbarend

\section{Static and Dynamic Code}\index{static code}\index{dynamic code}
\label{dynamic-section}

A \CLPR{} program is divided into static rules, which do not change,
and dynamic rules, which allow the rulebase to be modified via {\tt assert/1}
and {\tt retract/1} as well as by 
consulting\index{{\tt assert/1}}\index{{\tt retract/1}}.
As mentioned above, static rules/code cannot span more than one chunk.
\chgbarbegin
Dynamic code on the other hand can be defined anywhere and dynamic
rules can be added by asserting them during execution
or by consulting a program file, which behaves as if those definitions
were asserted.
The only requirement for rules intended to be dynamic is that the
particular predicate name has
to be pre-declared using {\tt dynamic/2} which ensures that
all uses of this predicate are now dynamic, e.g. {\tt ?- dynamic(foo, 2).}
The first argument is the name of the predicate and the second is its
arity\footnote{Most PROLOG's use the name/arity convention to specify
this but this could be confused with division, hence the two argument
form is used}.
Every dynamic declaration has to occur before any use of a dynamic predicate
is made (including {\tt rule}, {\tt assert} and {\tt retract}), otherwise
an error is generated with any of the preceeding system predicates and
any use of that predicate is assumed to be static.
\chgbarend
Declaring a predicate to be dynamic allows the use of {\tt rule/2}
to inspect the rulebase,
{\tt assert/1} to add new rules and {\tt retract/1} to delete rules.
\index{rule}\index{assert}\index{retract}

The operational semantics of the 
assert, rule and retract family of system
predicates is that any modifications to the rulebase
occur immediately and are immediately available for use\footnote{
The operational semantics of dynamic code may vary considerably
between different PROLOG systems hence one should not place
undue reliance on it.}.
This is called the immediate update view \cite{LINDHOLM87}.
Consider the following example: 

\stuff{
\>?- dynamic(p,0). \\
\>p :- assert(p), fail. \\
}

This will cause the goal ``{\tt ?- p.}'' to succeed.
Apart from the dynamic declaration and the immediate update semantics,
there is no difference between static and dynamic code and they
may be used interchangeably, e.g. both can be listed with {\tt ls/1}.
Dynamic code is also compiled but is generally not as efficient
as static code and also less deterministic.
Also note that the semantics of {\tt assert}, 
{\tt rule} and {\tt retract} are an enhancement
of that in PROLOG (see Section \ref{assert-section}).

\section{Debugging Support}\index{debugging}

The debugging facilities in this version of \CLPR\ are rudimentary.

\begin{description}

\ouritem{codegen\_debug}\index{{\tt codegen\_debug/0}}
This is a compiler directive, which includes debugging instructions in
subsequently generated code.
It should be active before the file to be debugged is consulted.

\ouritem{codegen\_nodebug}\index{{\tt codegen\_nodebug/0}}
This is a compiler directive that turns off the generation of debugging code in
subsequent compilation.

\ouritem{spy}\index{{\tt spy/0}}
This switch makes all relations compiled under {\tt codegen\_debug}
visible to the debugger. Protected rules are never visible.

\ouritem{spy(+P, +A)}\index{{\tt spy/2}}
This switch makes the relation for predicate {\tt
P} with arity {\tt A} visible to the debugger if it was compiled under {\tt
codegen\_debug}. It cannot be applied to
protected relations.

\chgbarbegin
\ouritem{spy({[P1(+A1),...,Pn(+An)]})}\index{{\tt spy/1}}
Like {\tt spy/2}, except a list is supplied of the predicates to
be spied on where the {\tt Pi}'s are the predicate names 
and the {\tt Ai}'s their arity.
\chgbarend

\ouritem{nospy}\index{{\tt nospy/0}}
Makes all relations invisible to the debugger.

\ouritem{nospy(+P, +A)}\index{{\tt nospy/2}}
Makes the relation for predicate {\tt P} with arity {\tt A} 
invisible to the debugger.

\chgbarbegin
\ouritem{nospy({[P1(+A1),...,Pn(+An)]})}\index{{\tt nospy/1}}
Like {\tt nospy/2}, except a list is supplied of the predicates to
be spied on where the {\tt Pi}'s are the predicate names and 
the {\tt Ai}'s their arity.
\chgbarend

\ouritem{trace}\index{{\tt trace/0}}
Activates printing. All subsequent attempts to search a relation visible
to the debugger will result in a message being printed. The message is the
same regardless of whether this is a first or subsequent attempt 
to satisfy a goal.

\ouritem{notrace}\index{{\tt notrace/0}}
De-activate printing.

\end{description}

\section{Notes on Efficiency}

Here we indicate some key features that can significantly 
affect efficiency.  Some of them are unsound in general, and hence
extreme care should be taken when using them.
Novice programmers may (and probably should) skip this section entirely.

\begin{itemize}
\item {\em Indexing} \index{indexing}\nl
	\CLPR\ employs first argument indexing for constructed functor 
	terms as well as real numbers.   Using indexing can result in
	significant speedups.
\item {\em Tail recursion} \index{tail recursion}\nl
	Last call optimization is employed, and hence procedures that
	are tail-recursive will not increase local stack usage.
\item {\em Logical disjunction ``{\tt ;/2}'' and 
	if-then-else ``{\tt ->/2}''} \index{disjunction} \index{if-then-else}\nl
	These are implemented at the meta-level and hence are not
    particularly efficient.
\item {\em Dynamic code} \index{dynamic code}\nl
	Dynamic code is slower than static code and is also less
	deterministic. Cuts can be used to make it more deterministic.
	Also, since dynamic code is compiled, asserting large terms
	may not be very fast.
\item {\em Garbage collection} \index{garbage collection}\nl
	Not implemented as yet.
\item {\em Implicit equalities} \index{implicit equalities}\nl
	The solving of inequalities that imply some implicit equations can
	be controlled using {\tt implicit/0}, {\tt noimplicit/0},
	{\tt partial\_implicit/0} (see Section \ref{special-section}).
\item {\em Asserting a rule} \index{fassert}\index{assert}\nl
	The predicate {\tt assert/1} involves incorporating the constraints
	that relate the variables in that rule (see Section \ref{assert-section}).
	This is less efficient than if the constraints were not taken
	into account. The {\tt fassert} family of special 
    predicates (``fast assert'') performs assertion without
    incorporating arithmetic constraints (see Section \ref{special-section}),
	as in PROLOG.
\end{itemize}

\section{Notes on Formal Correctness}

The following identifies the main reasons why the \CLPR{} implementation
does not perfectly conform to the idealized CLP scheme. 
\begin{itemize}
\item No occurs check during (functor) unification;
\item Depth-first search (loss of completeness);
\item Floating point:
because this implementation of \CLPR\ makes use of double precision floating
point arithmetic, some problems may be caused by artifacts such as roundoff.
The most common problem is that a constraint used as a test (in that all
variables are ground) unexpectedly fails because of round-off. This is dealt
with by adjusting the amount of slack that the system allows in numerical
comparisons, using the {\tt -z} command line option.
\item Nonlinear and meta-level constraints are delayed.
\end{itemize}

\chapter{Built-In Facilities}

\section{System Predicates}

\subsection{Rulebase}

\begin{description}
\ouritem{op(+P, +T, +S)}\index{{\tt op/3}}
Declares the atom {\tt S} to be an operator of type {\tt T} 
with precedence {\tt P}.
The type can be used to specify prefix, postfix and binary operators
using the positional notation: {\tt fy, fx, yf, xf, yfy, xfy, yfx, xfx};
where the ``{\tt f}'' specifies the operator and 
the ``{\tt y}'' and ``{\tt x}'' the arguments.
A ``{\tt y}'' specifies that the topmost functor/operator in the subexpression
be of the same or lower precedence than the operator ``{\tt f}'', 
and ``{\tt x}'' specifies that it is to be strictly lower.
The precedences must range between $\{0 \ldots 1200\}$.
where a 0 precedence removes the operator.

(See also Section \ref{op-section} for some examples.)

\oursitem{listing}\index{{\tt listing/0}}

\ouritem{ls}\index{{\tt ls/0}}
List the rules of the entire rulebase that are currently visible.

\oursitem{listing +P}\index{{\tt listing/1}}

\ouritem{ls +P}\index{{\tt ls/1}}
List the currently visible rules for the predicate {\tt P}, of all arities.

\oursitem{consult(+F)}\index{{\tt consult/1}}

\ouritem{{[+F]}}\index{{\tt [{$\cdots$}]}}
Read the file {\tt F} and add rules that it contains to the database. Goals in
the file are handled in a way that is described in 
Section \ref{consult-section}.
If the filename is specified as {\tt user} then the standard input
is used instead of a file.
\chgbarbegin
The form {\tt [F]} takes a list of filenames while {\tt consult/1} takes
only a single file.
When the file {\tt F} cannot be read then a possible list of file suffixes
is added using the CLPRSUFFIX environment variable 
(see Section \ref{sec:filenames}).
By default, a ``.clpr'' file extension is used.
(Not currently implemented: If the
file has a {\tt .clam} extension it is expected to be clam code and is loaded
appropriately. If it has no extension and a version with a {\tt .clam}
extension exists it is given preference.)
\chgbarend

\oursitem{reconsult(+F)}\index{{\tt reconsult/1}}

\ouritem{{[`+F]}}\index{{\tt [`{$\cdots$}]}}
Same as {\tt consult}, but if a predicate already has rules
defining it from before, they are deleted before the new ones are added,
and a warning message is printed. Note
that {\tt [-F]}, which is a common synonym for {\tt reconsult} in PROLOG
systems, cannot be used (since it means negative {\tt F}).

\ouritem{retractall}\index{{\tt retractall/0}}
Delete entire unprotected portion of the rulebase.

\ouritem{retractall(+H)}\index{{\tt retractall/1}}
Delete all currently visible rules with heads matching {\tt H}.
Static code cannot be deleted with {\tt retractall/1}.

\ouritem{asserta(+R)}\index{{\tt asserta/1}}
Add rule {\tt R} to the rulebase before all others defining 
the same predicate.  
Note that coded terms become uncoded in the rulebase.
See Section \ref{assert-section} for more information on meta-coding
of rules and differences with the usual PROLOG semantics.

\oursitem{assertz(+R)}\index{{\tt assertz/1}}
\ouritem{assert(+R)}\index{{\tt assert/1}}
Add rule {\tt R} to the rulebase after all others defining the same predicate.
Note that coded terms become uncoded in the rulebase.
See Section \ref{assert-section} for more information on meta-coding
of rules and differences with the usual PROLOG semantics.

\ouritem{rule(+H,?B)}\index{{\tt rule/2}}
True if the rule {\tt H:-B} is in the currently visible part of the rulebase.
Finds the next matching rule on backtracking.  
Note that the rules in the rulebase are coded before
matching is done.
See Section \ref{assert-section} for more information on meta-coding
of rules and differences with the usual PROLOG semantics.

\ouritem{deny(+H,?B)}\index{{\tt deny/2}}
Delete rule matching {\tt H :- B} from the currently visible part of the
rulebase. Also tries again on backtracking.
It is similar to {\tt retract/1} and both {\tt H} and {\tt B} are
coded terms.
See Section \ref{assert-section} for more information on meta-coding
of rules and differences with the usual PROLOG semantics.

\ouritem{retract(+R)}\index{{\tt retract/1}}
Delete rule matching {\tt R} from the currently visible part of the rulebase.
Like {\tt rule/2}, this has a ``coded view'' of the rulebase.
See Section \ref{assert-section} for more information on meta-coding
of rules and differences with the usual PROLOG semantics.

\ouritem{prot(+P,+A)}\index{{\tt prot/2}}
Protect all rules for predicate {\tt P} with arity {\tt A} in the rulebase.
This makes them look like system predicates to the user. In particular, they
cannot be listed, asserted or retracted.

\ouritem{prot({[P1(+A1),...,Pn(+An)]})}\index{{\tt prot/1}}
Same effect as {\tt prot/2} described above, but takes a list of predicate
names {\tt Pi} with arities {\tt Ai} in parentheses.

\end{description}

\subsection{Control}

\begin{description}
\ouritem{!}\index{{\tt {\cut}/0}}
The dreaded cut. As usual, its use is not recommended. It is often more
appropriate to use {\tt once/1}.
\ouritem{fail}\index{{\tt fail/0}}
Always fails.
\ouritem{true}\index{{\tt true/0}}
Always succeeds.
\ouritem{repeat}\index{{\tt repeat/0}}
Always succeeds, even on backtracking.
\ouritem{+B1 , +B2}\index{{\tt ,/2}}
Logical conjunction.
\ouritem{+B1 ; +B2}\index{{\tt ;/2}}
Logical disjunction. A cut inside one of these will behave very strangely.
That is, it will behave as if the two sides of the ``{\tt ;}'' 
are separate rules.
\chgbarbegin
(Note that because {\tt ;/2} is currently implemented as a meta call
it may sometimes not behave as if it was defined using an auxiliary predicate.
This can occur if there is an arithmetic term that causes failure.
The following short example illustrates the difference between
{\tt try/3} and {\tt try1/3} for the goal {\tt ?- try(X, 1, 0)}, \\
\stuff{
\>try(X, Y, Z) :- X=Y/Z ; X=1. \\
\>try1(Y/Z,Y,Z). try1(1,Y,Z). 
}\\
This may possibly change to be the same in some future version.)
\chgbarend
\ouritem{+C -> +B1 ; +B2}\index{{\tt ->;/3}}
If {\tt C} then call {\tt B1} otherwise call {\tt B2}. Uses unsafe negation.
Inefficient, since it uses {\tt call/1}. A cut inside one of these will behave
very strangely.
\end{description}

\subsection{Meta Level}

\begin{description}

\ouritem{call(+X)}\index{{\tt call/1}}
Usual meta level call, behaving as if the predicate {\tt X} appeared
directly in the body of a rule or goal.
Note that this form must be used -- it is not
permissible to simply put a variable in the body of a rule. 
Both static and dynamic code can be used with {\tt call}.
In this version, a cut inside a {\tt call} is ignored.
Also, {\tt printf/2} and {\tt
dump/1} cannot be used inside {\tt call}.
Both these restrictions can be avoided by simply redefining them using
a subsidiary rule.
\ouritem{not(+X)}\index{{\tt call/1}}
Unsafe negation. It is implemented using {\tt call/1}, so it is also likely to
be rather slow.

\ouritem{dump(+L1, ?L2, ?L3)}\index{{\tt dump/3}}
Similar to {\tt dump/2} (see Section \ref{dump-preds}); 
the first argument {\tt L1} represents
the target variables and the second argument {\tt L2} represents
new variables.  The difference with {\tt dump/2} is that (a) the 
projection is meta-coded (cf. Section \ref{meta-section}), and 
(b) this projection is not output but rather constructed as
the third argument {\tt L3} (cf. Section
\ref{dump-preds}).
Note that {\tt dump/3} does change the current collection
of constraints.

\ouritem{once(+X)}\index{{\tt once/1}}
This is equivalent to {\tt call(X), !} and unfortunately right now it is
implemented that way as well. Only the first answer to the query {\tt X} is
considered.

\ouritem{nonground(?X)}\index{{\tt nonground/1}}
True if {\tt X} is not a ground term.

\ouritem{ground(?X)}\index{{\tt ground/1}}
True if {\tt X} is a ground term.

\ouritem{nonvar(?X)}\index{{\tt nonvar/1}}
True if {\tt X} is not a variable: i.e, it has been constructed or grounded.

\ouritem{var(?X)}\index{{\tt var/1}}
True if {\tt X} is a variable. It may have been involved in an arithmetic
constraint, but has not been grounded or constructed.

\ouritem{?X == ?Y}\index{{\tt ==/2}}
True if {\tt X} and {\tt Y} are bound to exactly the same term.  In
particular, variables in equivalent positions must be identical. 
For example {\tt ?- X == Y} fails while {\tt ?- X = Y, X == Y}
succeeds.

\ouritem{atom(?X)}\index{{\tt atom/1}}
True if {\tt X} is an atom --- that is, a functor constant (including
the empty list).

\ouritem{atomic(?X)}\index{{\tt atomic/1}}
True if {\tt X} is an atom or real number.

\ouritem{functor(?X)}\index{{\tt functor/1}}
True if {\tt X} is constructed with a functor.

\ouritem{real(?X)}\index{{\tt real/1}}
Enforces a constraint that {\tt X} can take real values;
it is equivalent to any tautologous arithmetic constraint
involving {\tt X}, eg: {\tt X + 0 = X}. 

\ouritem{arithmetic(?X)}\index{{\tt arithmetic/1}}
True if {\tt X} is constrained to have a real value.
Note that this is just a passive test, as opposed to {\tt real/1}.

\ouritem{?T =.. ?L}\index{{\tt =../2}}
{\tt T} is a term and {\tt L} is the term expanded as a list. 
(Also known as {\tt univ/2}).  This predicate can be used to both
decompose and construct terms.  For its use either 
the first argument must be constructed
(a nonvar), or the second argument
must be a list of fixed length whose first element is a functor constant.

\chgbarbegin
\ouritem{functor(?T, ?F, ?A)}\index{{\tt functor/3}}
{\tt T} is a term, {\tt F} and {\tt A} are the name and
arity of the principle functor of {\tt T}.  Either {\tt T} must be
constructed or {\tt F} must be a functor constant (not a real number)
and {\tt A} must be a nonnegative integer.

\ouritem{arg(+N, +T, ?A)}\index{{\tt arg/3}}
{\tt A} is the {\tt N}$^{th}$ argument of term {\tt T}.
{\tt N} must be a positive integer and {\tt T} a compound term.
If {\tt N} is out of range the call fails.
\chgbarend

\ouritem{occurs(-V,?T)}\index{{\tt occurs/2}}
{\tt V} is a variable occurring in term {\tt T}.

\ouritem{floor(+R, -I)}\index{{\tt floor/2}}
{\tt R} must be a real number, and
{\tt I} is the largest integer smaller than or equal to {\tt R}.

\ouritem{dynamic(+P,+A)}\index{{\tt dynamic/2}}
Declares the predicate {\tt P} with arity {\tt A} to be dynamic, 
so that rules
can be added and deleted at will.

\end{description}

\subsection{Input/Output}

In this section,
non-ground variables will either be printed with a specified name
(like that in the argument of {\tt dump/1}),
or if one is not specified they are printed in one of the following formats:

\begin{description}
\ouritem{\_h\%d} Heap variable.
\ouritem{\_s\%d} Local stack variable.
\ouritem{\_t\%d} Parametric variable in solver.
\ouritem{\_S\%d} Slack variable in solver.
\end{description}

Input/Output facilities are as follows.

\begin{description}

\ouritem{dump(+L)}\index{{\tt dump/1}}
List the collection of current constraints on the current output
stream, projected with respect to the target variables in the list {\tt L}.
The list {\tt L} must be explicitly supplied, that is, it is written
syntactically as the argument of {\tt dump}.
The ordering of variables in the list is used to represent
the priority of the target variables (see Section \ref{dump-preds}).

\ouritem{dump(+L1, +L2)}\index{{\tt dump/2}}
A more flexible version of {\tt dump/1}, without its syntactic restriction.
Its first argument {\tt L1} represents the target variables,
and its second argument {\tt L2}, which must be ground,
represents the new names to be used in the output.  
The elements of these two lists can be arbitrary terms.  
(See Section \ref{dump-preds} for further explanation.)
Note that {\tt dump/2} does not change the current
collection of constraints.

\ouritem{nl}\index{{\tt nl/0}}
Send a newline character to the current output stream.

\oursitem{print(?T)}\index{{\tt print/1}}

\ouritem{write(?T)}\index{{\tt write/1}}
Print the term {\tt T}, according to {\tt op} declarations, on the current
output stream.

\ouritem{writeln(?T)}\index{{\tt writeln/1}}
The same as {\tt write(T), nl}.

\ouritem{printf(+F,+L)}\index{{\tt printf/2}}
Print the terms in the list {\tt L} on the current output stream in the format
given by the string {\tt F}. The behavior is similar to the {\tt printf}
library function in C. Every character except for the special escape
or argument patterns will be printed unchanged on the output.
The special escape characters begin with a ``\ttsl'' and are: 

\begin{center}
\begin{tabular}{|l|l|}
\hline
{\ttsl{\it XXX}} & the character represented by the octal number {\it XXX} \\
{\tt \ttsl{}n} & a new line \\
{\tt \ttsl{}r} & carriage return \\
{\tt \ttsl{}b} & backspace \\
{\tt \ttsl{}f} & form feed \\
{\tt \ttsl{}{\it X}} & any other character {\it X} appears unchanged \\
\hline
\end{tabular}
\end{center}

The argument patterns all begin with ``{\tt \%}'' and are used to
denote the formatting for each of corresponding terms in the list {\tt L}.
A ``{\tt \%\%}'' denotes a single percent.
Otherwise the format takes the form of an optional field width and optional
precision followed by one of the C printf conversion characters.
More precisely this can be described with the regular expression:

\stuff{\>\%[[-][0-9]*][\ttsl.[0-9]*][fegdoxcus\%]}

\chgbarbegin
The integral specifiers will print the real number, which
has been rounded to an integer using the ``even''
rounding rule.
\chgbarend
An empty list is needed if no variables are to be printed.
As a convenience, a single ``{\tt \%}'' may be used instead of
a specific argument format and a default format appropriate to
that particular argument will be used (with numbers
the default is printf format ``{\tt \%g}''). For example,

\stuff{
\> printf("X = \% Y =\%3.2g{\ttsl}n", [X, Y]).
}

\chgbarbegin
\ouritem{printf\_to\_atom(?A, +F, +L)}\index{{\tt printf\_to\_atom/3}}
Like {\tt printf/2} except that instead of being printed {\tt A}
is equated with an atom whose string is the same as what would otherwise
be printed.
\chgbarend

\ouritem{read(-X)}\index{{\tt read/1}}
Read a term from the current input and bind the variable X to it. Any
variables in the input term are deemed to be disjoint from variables appearing
in the rule. If an end of file is read, the term {\tt ?-(end)} is returned.
Finally, the term obtained is in quoted form. 
That is, any arithmetic operators
are treated syntactically. 

\ouritem{see(+F)}\index{{\tt see/1}}
Make {\tt F} the current input file.

\ouritem{seeing(?F)}\index{{\tt seeing/1}}
True when {\tt F} is the current input file.

\ouritem{seen}\index{{\tt seen/0}}
Close current input file. Revert to ``user'' (standard input).

\ouritem{tell(+F)}\index{{\tt tell/1}}
Make {\tt F} the current output file.

\ouritem{telling(?F)}\index{{\tt telling/1}}
True when F is the current output file.

\ouritem{told}\index{{\tt told/0}}
Close current output file. Revert to ``user'' (standard output).

\ouritem{flush}\index{{\tt flush/0}}
Flush the buffer associated with the current output file.
\end{description}

\subsection{Unix-Related Facilities}

\begin{description}

\ouritem{fork}\index{{\tt fork/0}}
Split the current process. Fails in one child and succeeds in the other.
\chgbarbegin
Not available under MS/DOS\footnote{MS/DOS is a trademark of Microsoft
Corporation} and OS/2.
\footnote{OS/2 is a trademark of IBM corporation}
\chgbarend

\ouritem{pipe(+X)}\index{{\tt pipe/1}}
Create a pipe named {\tt X}. For use with {\tt see}, {\tt tell}, etc.
\chgbarbegin
Not available under MS/DOS or OS/2.
\chgbarend

\ouritem{edit(+F)}\index{{\tt edit/1}}
Invoke the default editor on file {\tt F}, and then reconsult the file.
\chgbarbegin
Under UNIX\footnote{UNIX is a trademark of Bell Laboratories.}
the default is ``{\tt vi}'', under MS/DOS and OS/2 it is ``{\tt edit}''.
If the environment variable EDITOR is set then that is used instead.
\chgbarend

\ouritem{more(+F)}\index{{\tt more/1}}
Run the file {\tt F} through the ``{\tt more}'' utility or 
what the environment variable PAGER
has been set to.

\ouritem{halt}\index{{\tt halt/0}}
Exit from the \CLPR\ system.

\ouritem{clpr}\index{{\tt clpr/0}}
True.  Used to test if the program is executing in the \CLPR{} system.

\ouritem{abort}\index{{\tt abort/0}}
Abort execution of the current goal.

\ouritem{sh}\index{{\tt sh/0}}
Invoke an image of ``{\tt sh}'' on UNIX systems.
\chgbarbegin
On MS/DOS or OS/2, starts a sub-shell of ``{\tt command.com}'' or what
the environment variable COMPSEC as been set to.
\chgbarend

\ouritem{csh}\index{{\tt csh/0}}
Invoke an image of ``{\tt csh}'' under UNIX systems.
\chgbarbegin
On MS/DOS or OS/2 behaves the same as {\tt sh/0}.
\chgbarend

\ouritem{oracle(+F,+P1,+P2)}\index{{\tt oracle/3}}
Run the executable binary file {\tt F} and set up a pipe {\tt P1} for writing
to the process and a pipe {\tt P2} for reading from the process. These pipes
will be attached to the processes standard input and standard output
respectively.
\chgbarbegin
Not available on MS/DOS or OS/2.
\chgbarend


\end{description}

\subsection{Miscellaneous Facilities}

\begin{description}

\ouritem{history}\index{{\tt history/0}}
Print last 50 command line goals.

\chgbarbegin
\ouritem{history +N}\index{{\tt history/1}}
Print last {\tt N} command line goals.
\chgbarend

\ouritem{h}\index{{\tt h/0}}
Short for {\tt history/0}.

\ouritem{N}
Run the command line goal at position {\tt N} in the history list.
This may only be used as toplevel command.

\ouritem{new\_constant(+A, +N)} \index{{\tt new\_constant}}
\index{{\tt symbolic constants}}
\chgbarbegin
Sets the numeric symbolic constant {\tt A} to the value {\tt N}.
The constant name is specified without a ``{\tt \#}'',
e.g. {\tt ?- new\_constant(my\_constant, 5)}.
A warning is printed if the value of a known constant is changed
and the warning can be turned off with {\tt warning(warning\_off)}.
\chgbarend

\ouritem{srand(+X)}\index{{\tt srand/1}}
Set random number seed to the real number X.

\ouritem{rand(-X)}\index{{\tt rand/1}}
Generate uniformly distributed random number 0 and 1 inclusive and bind it to
{\tt X}. The quality of the routine used is not guaranteed. 

\ouritem{ztime}\index{{\tt ztime/0}}\index{statistics}
Zero the CPU time counter.

\ouritem{ctime(-T)}\index{{\tt ctime/0}}\index{statistics}
Binds {\tt T} to the elapsed CPU 
time since the counter was last zeroed. {\tt T}
should have been uninstantiated.

\chgbarbegin
\ouritem{style\_check(+A)}\index{{\tt style\_check/1}}
\index{contiguous}\index{singleton variable}
\index{{\tt single\_var}}\index{{\tt discontiguous}}
\index{{\tt name\_overload}}
Style checking warns about possible program errors. It is to be used
with {\tt A} being one of {\tt single\_var}, {\tt discontiguous},
{\tt name\_overload} and {\tt all}.
A warning is given when the style check rule is violated.
The option {\tt all} turns on both the checks.
The special option {\tt reset\_all} clears all previous pending
warnings which may have accumulated if style checking was off
and turns on full style checking.
See Section \ref{style-section} for details.

\ouritem{no\_style\_check(+A)}\index{{\tt no\_style\_check/1}}
\index{contiguous}\index{singleton variable}
\index{{\tt single\_var}}\index{{\tt discontiguous}}
\index{{\tt name\_overload}}\index{{\tt reset\_all}}
The reverse of {\tt style\_check/1} and turns off the corresponding options
{\tt single\_var}, {\tt discontiguous}, {\tt name\_overload} and {\tt all}.

\ouritem{\$clear\_style\_check}\index{{\tt \$clear\_style\_check}}
Clears any pending old style check warnings that may occur when
style checking is turned from off to on. Usually it is reasonable
not to need to use this and this is more meant for special uses.

\ouritem{warning(+A)}\index{{\tt warning/1}}\index{warning}
\index{abort}\index{errors}\index{continue}\index{warning\_on}
\index{warning\_off}\index{redefine\_on}\index{redefine\_off}
The behavior when an error occurs can be modified with
{\tt warning/1}. By default, when an error occurs a warning
error message is printed and execution is aborted back to the top level.
The various options for warning change this behavior.
The options for {\tt A} must be one of 
{\tt abort}, {\tt continue},
{\tt warning\_on}, {\tt warning\_off}, {\tt redefine\_on}
and {\tt redefine\_off}.
The options {\tt continue} or {\tt abort} control whether or not
execution is aborted back to the top level on an error.
The printing of warning messages is controlled by 
{\tt warning\_on} and {\tt warning\_off},
while {\tt redefine\_on} and {\tt redefine\_off} control whether or not
redefinitions of predicates during a reconsult issue a warning.
The option {\tt abort} overrides {\tt warning\_on} and warning messages
are displayed when {\tt abort} is active.
Otherwise the paired options here behave indepently.
\chgbarend
\end{description}

\subsection{Special Facilities}
\label{special-section}
These are unsupported 
facilities which may be used to gain more efficiency
under certain circumstances or are experimental in nature.
They should be used with care and may change or disappear.

\begin{description}
\ouritem{fassert(+R)}\index{{\tt fassert/1}}
Like {\tt assert/1} but it does not take into account meta-level
constraints or arithmetic constraints and is like assert in PROLOG.
Consequently it is faster than {\tt assert/1} but makes less sense
when there are constraints involved.
When the rules are ground, fassert behaves the same as assert.

\oursitem{fasserta(+R)} \index{{\tt fasserta/1}} 
\ouritem{fassertz(+R)} \index{{\tt fassertz/1}}
Ditto for {\tt asserta/1} and {{\tt assertz/1}}.

\chgbarbegin
\ouritem{\$call(+X)} \index{{\tt \$call/1}}
Meta level call on a single user-defined predicate only.
No compound goals or system predicates are allowed.
\chgbarend

\ouritem{implicit} \index{{\tt implicit/0}}
Implicit equalities are detected. This is the default.
A set of inequalities can sometimes be equivalent to some equations;
and these are known as implicit equalities. A trivial example of an
implicit equation is the following:

\stuff{
\>X >= 0, X <= 0 {\rm ~~~is equivalent to~~~} X = 0.
}

The {\tt implicit/0} flag controls whether these implicit equations
are detected by the constraint solver.
One caveat to note with the use of these flags is that switching them
on or off should be applied betweem different goal executions and not
during an actual execution.
Another important point is that, when there are nonlinear constraints,
turning off implicit equations may lead to delayed constraints
not being awakened.

\ouritem{noimplicit} \index{{\tt noimplicit/0}}
Turns off detection of implicit equalities. These are equations
which are implied by the collection of inequality constraints.
The implication of this is that delayed constraints which would otherwise
be awakened may continue to be delayed instead.
Constraint solving may or may not be faster with {\tt noimplicit}.

\ouritem{partial\_implicit} \index{{\tt partial\_implicit/0}}
Detects only some implicit equalities. This may be
faster than {\tt implicit}.

\chgbarbegin
\ouritem{set\_counter(+C, +V)} \index{{\tt set\_counter/2}}\index{counter}
This is a global counter which is not changed by backtracking.
Sets the counter with the atomic name {\tt C} to the real number
value {\tt V}. The counter name can be any atomic name.

\ouritem{counter\_value(+C, ?V)} \index{{\tt counter\_value/2}}
{\tt V} is equated with the value of counter {\tt C}.

\ouritem{add\_counter(+C, +V)} \index{{\tt add\_counter/2}}
The counter {\tt C} is incremented by {\tt V}.
\chgbarend
\end{description}

% \subsection{Missing Predicates}
% 
% \begin{description}
% 
% \ouritem{compile(+F1,+F2)}\index{{\tt compile/2}}
% Invoke the compiler on file {\tt F1} producing the {\tt clam} file {\tt F2}.
% 
% \ouritem{display(?T)}\index{{\tt display/1}}
% Print {\tt T} in prefix format on current output.
% 
% \ouritem{printrule(+R)}\index{{\tt printrule/1}}
% Print the rule {\tt R} in a suitable format on the current output.
% 
% \ouritem{printgoal(+G)}\index{{\tt printgoal/1}}
% Print the goal {\tt G} in a suitable format on the current output.
% 
% \ouritem{eof(+X)}\index{{\tt eof/1}}
% True when {\tt X} is the term signifying the end-of-file condition.
% Actually, true when {\tt X = ?-(end)} but the predicate should be used for
% portability.
% 
% \ouritem{libdir ?D}\index{{\tt libdir/1}}
% Binds {\tt D} to the name of the directory that library files live in.
% 
% \ouritem{lib +F}\index{{\tt lib/1}}
% Consults the file {\tt F} from the library directory.
% \ouritem{true}\index{{\tt true/0}}
% Always succeeds.
% 
% \ouritem{abort}\index{{\tt abort/0}}
% Stop executing the current query and return to the command line.
% 
% \ouritem{int(+X)}\index{{\tt int/1}}
% True if {\tt X} is an integer.
% 
% \ouritem{string(+X)}\index{{\tt string/1}}
% True if X is a string. Not appropriate now that strings don't exist.
% 
% \ouritem{funstr(?F,?S)}\index{{\tt funstr/2}}
% Originally converted between a constant {\tt F} and a string {\tt S}. 
% 
% \ouritem{shell(+X)}\index{{\tt shell/1}}
% Invoke the default shell with the string {\tt X} as input.
% 
% \ouritem{clpr}\index{{\tt clpr/0}}
% Check if we're in heaven.
% 
% \end{description}

\section{Nonlinear and Delayed Constraints}
\label{nl-section}\index{delayed constraint}
\index{nonlinear constraint}\index{Out of range errors}
\index{{\tt */2}}\index{{\tt sin/1}}\index{{\tt arcsin/1}}\index{{\tt cos/1}}
\index{{\tt arccos/1}}\index{{\tt pow/2}}
\index{{\tt max/2}}\index{{\tt min/2}}\index{{\tt abs/1}}

\noindent
\chgbarbegin
This section describes the form of the 
delaying conditions for examples of the various nonlinear
constraints given below.
In some of the functions below, {\tt sin}, {\tt arcsin},
{\tt cos}, {\tt arccos}, there will be values of {\tt X} and {\tt Z}
which fall outside the range of that function.
Such invalid values will cause the constraint to fail and by default
a ``Out of range'' value is generated. See {\tt warning/1}.
\chgbarend

\begin{description}
\ouritem{Z = X * Y}
	Delays until {\tt X} or {\tt Y} is ground.
\ouritem{Z = sin(X)}
	Delays until {\tt X} is ground.
\chgbarbegin
\ouritem{Z = arcsin(X)}
	Delays until {\tt X} or {\tt Z} is ground.
\chgbarend
\ouritem{Z = cos(X)}
	Delays until {\tt X} is ground.
\chgbarbegin
\ouritem{Z = arccos(X)}
	Delays until {\tt X} or {\tt Z} is ground.
\chgbarend
\ouritem{Z = pow(X, Y)}
	Delays until (a) {\tt X} and {\tt Y} are ground, or 
	(b) {\tt X} and {\tt Z} are ground, \\
	or (c) {\tt X} = 1, or (d) {\tt Y} = 0, or (e) {\tt Y} = 1.
\ouritem{Z = abs(X)}
	Delays until (a) {\tt X} is ground, or (b) {\tt Z} = 0, or (c) {\tt Z} is ground
	and negative.
\ouritem{Z = min(X, Y)}
	Delays until {\tt X} and {\tt Y} are ground.\\
	(A proper implementation, delaying until {\tt X} $\leq$ {\tt Y} 
    or {\tt X} $\geq$ {\tt Y},
	may come later.)
\ouritem{Z = max(X, Y)}
	Similar to the above.
\ouritem{Z = eval(X)}
	Delays until {\tt X} is constructed.
\end{description}

\section{Pre-Defined Operators}\index{pre-defined operators}\index{operators}
\label{op-section}
\begin{verbatim}
::- op(21, fy, '-').
::- op(21, yfx, *).
::- op(21, yfx, /).
::- op(31, yfx, (-)).
::- op(31, yfy, +).
::- op(37, xfx, <).
::- op(37, xfx, <=).
::- op(37, xfx, >).
::- op(37, xfx, >=).
::- op(40, xfx, =).
::- op(40, xfx, =..).
::- op(40, xfx, is).
::- op(50, fx, `).
::- op(51, xfy, (.)).
::- op(60, fx, alisting).
::- op(60, fx, als).
::- op(60, fx, h).
::- op(60, fx, history).
::- op(60, fx, lib).
::- op(60, fx, libdir).
::- op(60, fx, listing).
::- op(60, fx, ls).
::- op(60, fx, not).
::- op(60, fx, once).
::- op(252, xfy, ',').
::- op(253, xfy, ;).
::- op(254, xfy, (->)).
::- op(255, fx, (:-)).
::- op(255, fx, (::-)).
::- op(255, fx, (?-)).
::- op(255, xfx, (:-)).
\end{verbatim}

\chapter{Installation Guide}
\label{installation-guide}
\index{installation guide}

Here we discuss how \CLPR{} can be made to run on a particular computer system.
\chgbarbegin
For installation details on MS/DOS or OS/2, please refer to the
appropriate README in the DOS directory.
\chgbarend

\section{Portability}\index{portability}

This version of compiled \CLPR{} should be easily portable to 32-bit 
computers running some reasonable variant of the UNIX operating system.
In most cases, all that will be necessary is for the installer to edit the
{\tt Makefile} to specify the machine and operating system, choose the C 
compiler, optimization level and name of the \CLPR{} executable file, and
run {\tt make}. 

\subsection{Pre-defined Installation Options}\index{installation options}

The {\tt Makefile} for \CLPR{} contains definitions of the environment 
variables {\tt CC}, {\tt CFLAGS}, {\tt EXEC} and {\tt OPTIONS}. They
should be checked before installation and adjusted as follows. 

\begin{itemize}

\item {\tt CC} is just the name of the C compiler to be used to compile the
\CLPR{} system. It is almost always reasonable to leave this as {\tt cc}, 
although many machines now have more efficient (and more correct) C compilers 
available. Information about these can be obtained from your system 
administrator.

\item {\tt CFLAGS} specifies switches of the above C compiler that need to 
be used. While various C compilers have their own range of switches that might
have to be used to make such a large program compile and run, in most
cases only the optimization level will be needed here. This will almost 
always be {\tt -O} but higher optimization levels may be available.
Also special flags may sometimes be necessary to utilize the
full performance of the native floating point hardware. 
However, it is important to realize that {\em many} 
optimizing C compilers have bugs that are only triggered by compiling 
a large program at a high optimization level. For this reason, the first 
attempt to install \CLPR{} should be made without invoking the C compiler's 
global optimizer. This usually involves just leaving the {\tt CFLAGS} field 
blank.

\item {\tt EXEC} specifies the name of the \CLPR{} binary to be generated. 
We recommend {\tt clpr}.

\item {\tt LIBPATH} specifies the default directory for the startup file
{\tt init.clpr}. It should be set to the directory in which \CLPR{} is
installed.

\item {\tt OPTIONS} is used to specify the hardware and operating system. 
A number of predefined options are available, which often need to be used 
in combinations.
\begin{enumerate}

\item[{\tt BSD}] Always set if system is running Berkeley Unix, or MACH,
or Ultrix. This is also to be set for NEXT machines.

\item[{\tt AIX}] Always set if system is running IBM's AIX operating system.

\item[{\tt SYS5}] This broadly indicates that some version of System V Unix
is being used. Should also be used in combination with {\tt AIX} flag and
if the operating system is Hewlett-Packard's HP/UX.

\item[{\tt IBMRT}] This indicates that the system is an IBM RT/PC.

\item[{\tt RS6000}] This indicates that the system is an IBM RS/6000.

\item[{\tt HP835}] This is needed for the Hewlett-Packard RISC workstations --
especially the 9000 series model 835. Note that it is not appropriate for 
those HP workstations based on Motorola processors.

\item[{\tt MIPS}] Needed for machines with MIPS CPU, such as SGI machines
and the DECStation 3100.

\chgbarbegin
\item[{\tt MSDOS}] Set for 386 or 486 PC's running MS/DOS.
See the file ``DOS/README.DOS'' for details.

\item[{\tt OS2V2}] Set for 386 or 486 PC's running OS/2 2.0 
(IMPORTANT NOTE: \CLPR{} version 1.2 will not
work on OS/2 1.x because that only supports 16-bit addressing).
See the file ``DOS/README.OS2'' for details.
\chgbarend
\end{enumerate}
So, for example an IBM RS/6000 running AIX would need the definition
\[\mbox{\tt OPTIONS = -DAIX -DSYS5 -DRS6000}\]
\end{itemize}

\chgbarbegin
One parameter which may have to be changed to ensure that the CPU timing
is correct on machines running System V Unix is the Hertz rate which
determines the unit of time measured.
Typically this value of {\tt HZ} is either 60 or 100.
The default value is 60 and otherwise it should be added to the
OPTIONS line in the Makefile, eg. -DHZ=100,
(on the RS/6000 the default is 100hz).
\chgbarend

\subsection{Customized Installation}


When \CLPR{} starts up it performs some consistency checks on some
of the default values in the startup.
In particular, on failure to startup it may recommend that
the definition of {\tt PMASK} be changed in {\tt emul.h}.
If that does not work or if a fatal installation error was reported
then you may have an unusual operating system problem which cannot
be easily fixed by the installer, and it may be best to 
contact the authors.

While there are various system limits, these are mostly parameterized
and can be changed either directly on the command line or by
recompiling with new values for the limits.\index{system parameters}
Most of the limits are contained in the file {\tt config.h},
and some of them will be described below.
The parameters which are not listed below may be more dangerous
to change arbitrarily.

\vspace*{5mm}
\noindent
\begin{tabular}{|l|l|}
\hline
Pre-defined constant & Meaning \\ \hline
DEF\_CLP\_SUFFIX & default suffix for \CLPR{} program files\index{suffix} \\ \hline
INITFNAME & default bootstrap file \index{bootstrap}\index{init.clpr} \\ \hline
DOS\_TMP\_FILE & name of temporary file used only under MSDOS or OS2 \\ \hline
DEFAULT\_EPS & default value of for zero \index{zero} \\ \hline
DEF\_CODE\_SZ & default maximum size of code space \\ \hline
MAX\_GOAL\_CODE & default maximum size of code for a top-level goal \\ \hline
DEF\_LSTACK\_SZ & default maximum stack size \\ \hline
DEF\_HEAP\_SZ & default maximum heap size \\ \hline
DEF\_TRAIL\_SZ & default maximum trail size \\ \hline
DEF\_SOLVER\_SZ & default maximum number of solver variables \\ \hline
MAX\_DUMP\_VAR & maximum number of variables for dump \\ \hline
MAX\_PROJ\_NUM & maximum number of real constants in dump \\ \hline
MAX\_IMPLICIT & maximum number of implicit 
	equations detected by a constraint \\
\hline
\end{tabular}
\vspace*{5mm}

\noindent
The stack, code, heap and trail sizes; value of zero; and the number of
solver variables can all be changed from the command line (see Section
\ref{command-section}).

\section{Basic Configuration}

The only file \CLPR{} system needs to read on startup is {\tt init.clpr}. 
\index{init.clpr}
It always looks for this file in the directory specified at runtime by the
environment variable {\tt CLPRLIB}, defaulting to either the current working 
directory or what {\tt LIBPATH} has been specified as in the {\tt Makefile}.
\index{CLPRLIB}\index{LIBPATH}

The only other environment variable which one may want to change is to
add your own list of file suffixes with the 
environment variable {\tt CLPRSUFFIX}.
The format is in the style of the UNIX {\tt sh} {\tt PATH} variable.

\chapter{Bug Reports and Other Comments}
\index{bug reports}

Please address all correspondence to

\begin{quote}
Joxan Jaffar, H1-D48 \\
IBM Thomas J. Watson Research Center \\
P.O. Box 704 \\
Yorktown Heights, NY 10598 \\
U.S.A. \\
{\tt (joxan@{watson.ibm.com, joxan@yktvmh.bitnet})}
\end{quote}

\newpage

\bibliographystyle{plain}
\bibliography{manual}

%\begin{theindex}
\begin{theindex}

  \item {\tt */2}, 55
  \item {\tt ,/2}, 47
  \item {\tt ->;/3}, 47
  \item {\tt ;/2}, 47
  \item {\tt =../2}, 49
  \item {\tt ==/2}, 48
  \item {\tt [`{$\cdots$}]}, 46
  \item {\tt [{$\cdots$}]}, 45
  \item {\tt \$call/1}, 54
  \item {\tt \$clear\_style\_check}, 54
  \item {\tt abort/0}, 52
  \item {\tt abs/1}, 13, 55
  \item {\tt add\_counter/2}, 55
  \item {\tt arccos/1}, 13, 55
  \item {\tt arcsin/1}, 13, 55
  \item {\tt arg/3}, 49
  \item {\tt arithmetic/1}, 49
  \item {\tt assert/1}, 18, 41, 46
  \item {\tt asserta/1}, 46
  \item {\tt assertz/1}, 46
  \item {\tt atom/1}, 48
  \item {\tt atomic/1}, 49
  \item {\tt call/1}, 48
  \item {\tt clpr/0}, 52
  \item {\tt codegen\_debug/0}, 42
  \item {\tt codegen\_nodebug/0}, 42
  \item {\tt consult/1}, 45
  \item {\tt cos/1}, 13, 55
  \item {\tt counter\_value/2}, 55
  \item {\tt csh/0}, 52
  \item {\tt ctime/0}, 53
  \item {\tt deny/2}, 46
  \item {\tt discontiguous}, 53, 54
  \item {\tt dump/1}, 23, 50
  \item {\tt dump/2}, 23, 50
  \item {\tt dump/3}, 23, 48
  \item {\tt dynamic/2}, 49
  \item {\tt edit/1}, 52
  \item {\tt eval/1}, 15
  \item {\tt fail/0}, 47
  \item {\tt fassert/1}, 54
  \item {\tt fasserta/1}, 54
  \item {\tt fassertz/1}, 54
  \item {\tt floor/2}, 49
  \item {\tt flush/0}, 52
  \item {\tt fork/0}, 52
  \item {\tt functor/1}, 49
  \item {\tt functor/3}, 49
  \item {\tt ground/1}, 48
  \item {\tt h/0}, 53
  \item {\tt halt/0}, 52
  \item {\tt history/0}, 53
  \item {\tt history/1}, 53
  \item {\tt implicit/0}, 55
  \item {\tt listing/0}, 45
  \item {\tt listing/1}, 45
  \item {\tt ls/0}, 45
  \item {\tt ls/1}, 45
  \item {\tt max/2}, 13, 55
  \item {\tt min/2}, 13, 55
  \item {\tt more/1}, 52
  \item {\tt name\_overload}, 53, 54
  \item {\tt new\_constant}, 53
  \item {\tt nl/0}, 50
  \item {\tt no\_style\_check/1}, 54
  \item {\tt noimplicit/0}, 55
  \item {\tt nonground/1}, 48
  \item {\tt nonvar/1}, 48
  \item {\tt nospy/0}, 43
  \item {\tt nospy/1}, 43
  \item {\tt nospy/2}, 43
  \item {\tt notrace/0}, 43
  \item {\tt occurs/2}, 49
  \item {\tt once/1}, 48
  \item {\tt op/3}, 45
  \item {\tt oracle/3}, 53
  \item {\tt partial\_implicit/0}, 55
  \item {\tt pipe/1}, 52
  \item {\tt pow/2}, 13, 55
  \item {\tt print/1}, 50
  \item {\tt printf/2}, 50
  \item {\tt printf\_to\_atom/3}, 51
  \item {\tt prot/1}, 47
  \item {\tt prot/2}, 47
  \item {\tt quote/1}, 15
  \item {\tt rand/1}, 53
  \item {\tt read/1}, 51
  \item {\tt real/1}, 49
  \item {\tt reconsult/1}, 46
  \item {\tt repeat/0}, 47
  \item {\tt reset\_all}, 54
  \item {\tt retract/1}, 18, 41, 46
  \item {\tt retractall/0}, 46
  \item {\tt retractall/1}, 46
  \item {\tt rule/2}, 18, 46
  \item {\tt see/1}, 51
  \item {\tt seeing/1}, 51
  \item {\tt seen/0}, 51
  \item {\tt set\_counter/2}, 55
  \item {\tt sh/0}, 52
  \item {\tt sin/1}, 13, 55
  \item {\tt single\_var}, 53, 54
  \item {\tt spy/0}, 42
  \item {\tt spy/1}, 43
  \item {\tt spy/2}, 42
  \item {\tt srand/1}, 53
  \item {\tt style\_check/1}, 53
  \item {\tt symbolic constants}, 53
  \item {\tt tell/1}, 51
  \item {\tt telling/1}, 51
  \item {\tt told/0}, 52
  \item {\tt trace/0}, 43
  \item {\tt true/0}, 47
  \item {\tt var/1}, 48
  \item {\tt warning/1}, 54
  \item {\tt write/1}, 50
  \item {\tt writeln/1}, 50
  \item {\tt ztime/0}, 53
  \item {\tt {\cut}/0}, 47

  \indexspace

  \item abort, 54
  \item analog to clause, 18
  \item analog to retract, 18
  \item answer constraints, 33
  \item arithmetic constraint, 6
  \item arithmetic term, 5
  \item assert, 42, 44

  \indexspace

  \item bootstrap, 60
  \item bug reports, 62

  \indexspace

  \item clam files, 31
  \item clause, 18
  \item CLPRLIB, 31, 61
  \item CLPRSUFFIX, 32
  \item code space, 32
  \item command line arguments, 32
  \item comments, 3
  \item constraint, 6
  \item consulted files, 40
  \item contiguous, 53, 54
  \item continue, 54
  \item counter, 55

  \indexspace

  \item debugging, 42
  \item delayed constraint, 12, 13, 33, 55
  \item disjunction, 43
  \item dump, 20, 23
  \item dynamic code, 41, 43

  \indexspace

  \item Environment variables, 31, 32
  \item errors, 54
  \item eval, 16, 17

  \indexspace

  \item fassert, 44
  \item file names, 32
  \item functor constraint, 6
  \item functor term, 5

  \indexspace

  \item garbage collection, 44
  \item goal, 3
  \item goals, 33, 41

  \indexspace

  \item heap, 32

  \indexspace

  \item if-then-else, 43
  \item implicit dump, 33
  \item implicit equalities, 44
  \item indexing, 43
  \item init.clpr, 31, 60, 61
  \item installation guide, 58
  \item installation options, 58

  \indexspace

  \item LIBPATH, 61
  \item local stack, 32

  \indexspace

  \item meta-programming, 15
  \item Monash interpreter, 69

  \indexspace

  \item nonlinear constraint, 12, 33, 55
  \item notation conventions, 2

  \indexspace

  \item operational model, 13
  \item operators, 56
  \item Out of range errors, 55
  \item output, 20

  \indexspace

  \item portability, 58
  \item pre-defined operators, 56
  \item program, 3
  \item projection, 20
  \item prot, 41

  \indexspace

  \item queries, 33
  \item quote, 15, 17

  \indexspace

  \item random number seed, 32
  \item redefine\_off, 54
  \item redefine\_on, 54
  \item retract, 42
  \item rule, 3, 42

  \indexspace

  \item sample session, 35
  \item singleton variable, 53, 54
  \item solver variables, 32
  \item static code, 41
  \item statistics, 53
  \item style checking, 34
  \item suffix, 32, 60
  \item system parameters, 60

  \indexspace

  \item tail recursion, 43
  \item target variables, 20, 23
  \item trail, 32
  \item type, 8

  \indexspace

  \item warning, 34, 54
  \item warning\_off, 54
  \item warning\_on, 54

  \indexspace

  \item zero, 32, 60

\end{theindex}

%\end{theindex}

\appendix
\chapter{Differences from the Monash Interpreter}\index{Monash interpreter}

Here we only list those facilities from the Monash interpreter that are
not supported.

\begin{itemize}
\item The issue of string handling has not yet been settled.
\item Predicate definitions cannot be indiscriminately spread over a
number of files.
\item There is no automatic variable generation for answer projection;
there is no {\tt dump/0} predicate.
\item Goals result in a prompt for alternate solutions whenever there is a
choice point, regardless of whether there are variables in the goal. 
\item No profiling; the predicates {\tt prof/0} and {\tt noprof/0} are
not available.
\item Sytem warnings are controlled differently using {\tt warning/1}.
\item Linear inequalities are always decided immediately rather than delayed;
the predicates {\tt ineq/0} and {\tt noineq/0} are not available.
\item Statistics are now available through special system predicates, so the
{\tt stats/0} system predicate, while may exist, is not supported.
\item There is no {\tt is/2} predicate.
\end{itemize}

\end{document}

